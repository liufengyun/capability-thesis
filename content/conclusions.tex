\section{Conclusion}

We formalized four capability-based effect systems and proved
soundness and effect safety for each system. The four systems can
serve as the theoretical foundation for implementing capability-based
effect systems in functional languages.

The existence of \emph{stoic functions} is the main trait of
capability-based effect systems. The interplay between \emph{stoic
  functions} and \emph{free functions} enables flexible programming
patterns that trivially solve the problem of \emph{effect
  polymorphism}.

Capability-based effect systems have to be paired with strict
evaluation, just like monad-based effect systems have to be paired
with lazy evaluation.

In the two proposed systems with universal types, namely F-Pure and
F-Impure, it's impossible to abstract over capability types and free
function types.

\subsection{Related Work}

Lucassen and Gifford first introduced type-and-effect
systems\cite{gifford1986integrating} and effect polymorphism using
effect type parameterization \cite{lucassen1988polymorphic}, which is
further developed by Talpin and Jouvelot to provide type-and-effect
inference \cite{talpin1992polymorphic,
  talpin1994type}. Type-and-effect inference can greatly reduce
verbosity in syntax, but it only works in languages with global type
inference, while Scala is based on local type inference. Even in those
languages with global type-and-effect inference, the type signature
for effect-polymorphic functions are much more complex than in
capability-based effect systems, which is an obstacle to programmers.

Moggi introduced the usage of monads for computation
effects\cite{moggi1991notions}. Wadler popularized the usage of
monads\cite{wadler1992comprehending, wadler1995monads} and proved that
it's possible to transpose any type-and-effect system into a
corresponding monad system\cite{wadler2003marriage}. As reported in
section 1.6 of the thesis\cite{lippmeier2009type}, almost every
general purpose higher-order function in Haskell needs both a monadic
version and non-monadic version. Lippmeier proposed the usage of
region variables and dependently kinded witness to encode mutability
polymorphism\cite{lippmeier2009witnessing}.

Lukas \emph{et al.}  studied type-and-effect systems for
Scala\cite{rytz2012lightweight, rytz2013flow, lukas2014effect}.  In
lightweight polymorphic effects\cite{rytz2012lightweight}, the
dichotomy between \emph{effect-polymorphic function type} and
\emph{monomorphic function type} resembles the dichotomy between
\emph{stoic function type} and \emph{free function type}. However, in
the system effect polymorphism doesn't work for nested functions. To
overcome this problem, they unified the two function types in a
framework called \emph{relative effect annotation} based on dependent
types. In the framework, the function $map$ can be marked as
effect-polymorphic without introducing a generic effect variable:

\begin{lstlisting}[language=Scala]
def map[A, B](f: A => B)(l: List[A]): List[B] @pure(f)
\end{lstlisting}

The annotation $@pure(f)$ says that the effect of $map$ depends on the
function $f$. Nested functions are handled specially in the system to
avoid duplicate annotations.

% Despite the syntactical improvements, it still incurs some notational
% burden on programmers.

Heather \emph{et al.} proposed a variant of functions called
\emph{spores} for Scala\cite{miller2014spores}. Compared to normal
functions, spores observe a variable capturing discipline. The set of
types that can or cannot be captured is part of the type signature of
spores, thus can be used by library authors to impose constraints on
parameters of function types.

% Spores are well suited for distributed and parallel computations, but
% don't generalize to a practical effect system due to the verbosity in
% syntax.

Crary \emph{et al.} proposed a capability calculus for typed memory
management\cite{crary1999typed}, which improves the LIFO-style
(last-in, first-out) region-based memory management introduced by
Tofte and Talpin\cite{tofte1997region} by allowing arbitrary
allocation and deallocation order. The safety of deallocation of
memory is guaranteed by the type system. In the capability calculus, a
capability is a set of regions that are presently valid to access. It
tracks uniqueness of capabilities in the type system to control
aliasing of capabilities. Functions may be polymorphic over types,
regions or capabilities. Capability variables cannot be captured in
closures and have to be passed around in the program.

\subsection{Future Work}

Objects and mutation are predominant features of industrial
languages. To provide a more practical model for industrial languages,
it's useful to extend existing systems with record types and mutation.

Effect masking can be a useful feature in real-world programming,
especially when dealing with exception effects. It's worthy to explore
effect masking in existing systems.

Though the effect polymorphism inherent in capability-based effect
systems reduces the need for parametric polymorphism over capability
types, parametric polymorphism over free function types is useful in
practical programming.  One approach to make parametric polymorphism
work for capability types and free function types is to integrate
\emph{bounded quantification} with capabilities. This is the what we
are working on now. We've managed to prove soundness of the system,
and the proof of effect safety is in progress.

% Another direction of work is to extend the proposed systems with some
% implicit calculus, in order to avoid explicitly passing capabilities
% around in the code. This would result in significant savings in
% boilerplate code, thus make the system more friendly to programmers.

% The most exciting work would be to implement a capability-based effect
% system in Scala or other industrial programming languages and
% popularize the usage of effect systems in the programming community.
