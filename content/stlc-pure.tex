\section{System STLC-Pure}

This chapter describes a variant of the \emph{simply typed
  lambda calculus} with the extension of capabilities. We call this
system \emph{STLC-Pure}, because in this system all functions must
observe a variable-capturing discipline.

% In later chapters, we'll see the development of impure systems where
% there are both effect-disciplined functions which observe the
% variable-capturing discipline, and ordinary functions which don't
% observe the variable-capturing discipline.

The system STLC-Pure, though conceptually simple, can quite well
demonstrate the main features of capability-based effect
systems. We'll first introduce the formalization, then discuss
soundness and effect safety. Concepts introduced here will be a
foundation for more complex systems in later chapters.

\subsection{Definitions}

Formally, STLC-Pure is obtained by introducing a capability type and
imposing a variable-capturing discipline on lambda abstractions.
Figure~\ref{fig:stlc-pure-definition} presents the full definition of
STLC-Pure.

The syntax is almost the same as standard STLC, except the addition of
the capability type \emph{E} and the taking of variables as
values. The evaluation rules are exactly the same, with standard
call-by-value small-step semantics. The typing rule \textsc{T-Abs} is
slightly changed by performing an operation \emph{pure} on the
environment. The peculiarities in the formalization are explained
below.

\begin{figure}[h]
\begin{framed}

% multi-column separator
\setlength{\columnseprule}{0.4pt}
\begin{multicols}{2}

\textbf{Syntax}

\begin{tabu} to \linewidth {l l l X[r]}
  t   & ::= &                    & terms:               \\
      &     &  x                 & variable             \\
      &     & $\lambda$ x:T.t    & abstraction          \\
      &     & t t                & application          \\
\\
  v   & ::= &                    & values:              \\
      &     & $\lambda$ x:T.t    & abstraction value    \\
      &     & \colorbox{shade}{x}& variable value       \\
\\
  T   & ::= &                    & types:               \\
      &     & B                  & basic type           \\
      &     & \colorbox{shade}{E}& capability type      \\
      &     & T $\to$ T          & type of functions    \\
\end{tabu}

\hfill\\

\textbf{Evaluation} \hfill \framebox[1.2\width][r]{$t \longrightarrow t'$}

\infrule[E-App1]
{ t_1 \longrightarrow t'_1 }
{ t_1 \; t_2 \longrightarrow t'_1 \; t_2 }

\infrule[E-App2]
{ t_2 \longrightarrow t'_2 }
{ v_1 \; t_2 \longrightarrow v_1 \; t'_2 }

\infax[E-AppAbs]
{ (\lambda x:T.t_1) v_2 \longrightarrow [x \mapsto v_2]t_1 }

\columnbreak

\textbf{Typing}  \hfill \framebox[1.2\width][r]{$\Gamma \vdash x : T$}

\infrule[T-Var]
{ x: T \in \Gamma }
{ \Gamma \vdash x : T }

\infrule[T-Abs]
{ \colorbox{shade}{$pure(\Gamma),\; x: S \vdash t_2 : T$} }
{ \colorbox{shade}{$\Gamma \vdash \lambda x:S.t_2 : S \to T$} }

\infrule[T-App]
{ \Gamma \vdash t_1 : S \to T \andalso \Gamma \vdash t_2 : S }
{ \Gamma \vdash t_1 \; t_2 : T }

\colorbox{shade}{\textbf{Pure Environment}}

\hfill

\begin{center}
\begin{tabular}{l c l}
pure($\varnothing$)             & = &   $\varnothing$ \\
pure($\Gamma$, x: E)            & = &  pure($\Gamma$) \\
pure($\Gamma$, x: T)  & = &  pure($\Gamma$), x: T     \\
\end{tabular}
\end{center}

\end{multicols}
\end{framed}

\caption{System STLC-Pure}
\label{fig:stlc-pure-definition}
\end{figure}

\subsubsection{Variable-Capturing Discipline}

The most important change to the standard STLC lies in the following
typing rule:

\infrule[T-Abs]
{ pure(\Gamma), \; x: S \vdash t_2 : T }
{ \Gamma \vdash \lambda x:S.t_2 : S \to T }

This typing rule imposes a \emph{variable-capturing discipline} on
lambda abstractions. This discipline stipulates that only variables
whose type is not a capability type can be captured in a lambda
abstraction.

The discipline is implemented with the helper function \emph{pure},
which removes all variable bindings of the capability type \emph{E}
from the typing environment. It's easy to verify that the function
\emph{pure} satisfies following properties:

\begin{lemma}[Pure-Distributivity]
  pure (E, F) = pure E, pure F
\end{lemma}

\begin{lemma}[Pure-Idempotency]
  pure (pure E) = pure E
\end{lemma}

Initially, we've tried a variable-capturing discipline where no free
variables can be captured, i.e. $pure(\Gamma) = \varnothing$. While
this definition is certainly effect safe, the system is not very
expressive, as we lose the ability to create closures in the system,
which is usually considered to be an essential feature of functional
programming.

We also tried to exclude variables of the type $E$ or $T \to E$ (for
any T) from the pure environment. As an over-approximation, this
version is certainly effect-safe. However, it rejects more types than
necessary, thus reduces expressiveness of the system.

\subsubsection{Stoic Functions and Free Functions}

The variable-capturing discipline makes the functions in STLC-Pure
different from functions in standard STLC. In STLC, functions can
capture any variables in scope, while in STLC-Pure functions can only
capture variables whose type is not the capability type E. To
differentiate them (which is important as in later systems both
exist), we call the more effect-disciplined functions \emph{stoic
  functions} (or stoics) and the other \emph{free functions}.

Stoic functions are essential in capability-based effect systems. If
functions are allowed to capture capability variables in scope, it
will be impossible to tell whether a function has side effect or not
(and what kind of effect) by just checking its type. Stoic functions
are effect-disciplined in the sense that the only way for stoic
functions to have side effects is to pass a capability as parameter,
thus it can be captured by the type system.

Stoic functions are not necessarily pure functions. Stoic functions
can have side effects, and if they do have side effects they are
honest about that in their type signature. For example, the following
function \emph{hello} is a stoic function with IO effects\footnote{For
  the sake of readability, we'll use a syntax similar to Scala. In
  particular, we'll use $\to$ for the type of stoic functions, and
  $\Rightarrow$ for free functions.}.

\begin{lstlisting}[language=Scala]
  def hello(c:IO) = println("hello, world!", c)
\end{lstlisting}

In the following code snippet, the function \emph{f} must be pure, as
it doesn't take any capability as parameter. The type system
guarantees that the function indeed cannot make any side effects.

\begin{lstlisting}[language=Scala]
  def twice(f: Int -> Int)(x: Int) = f (f x)
\end{lstlisting}

% In the functions passed, it's impossible to call functions to
% create side effects, as in STLC-Pure all functions that can have side
% effects take some capability as parameter.

\subsubsection{Where do Effects Come From}

There is no formalization of effects in current effect system. We
assume the existence of primitive functions like \emph{println} and
\emph{readln}, which take capabilities to make side effects.

\begin{lstlisting}[language=Scala]
  def println: String -> IO -> ()
  def readln: IO -> String
\end{lstlisting}


% As the primary concern of this study is IO effects,

\subsubsection{Where do Capabilities Come From}

It is impossible to create capabilities in current system. Where do
capabilities come from?  There are two possible answers: (1) all
capabilities are from the run-time and passed to the program through
the \emph{main} method; (2) there are no capabilities; they can be
erased before evaluation, without changing the meaning of programs.

% However, in this study we only state this observation informally and
% leave the formal proof to future studies.

\subsubsection{Why Treat Variables as Values}

As discussed above, there is no way to create a value of capabilities
explicitly. Thus, a function taking a parameter of the capability type
E can never be executed in the \emph{call-by-value} semantics, unless
variables are values. The same is true for the base type B.

Treating variables as values ensures that substitution of a term with
a value of the base type B or the capability type E can actually
happen in the system, thus makes the preservation proof more
convincing.

% Adding variables as values doesn't break soundness or effect safety of
% the system. In fact, in adding variables as values, we only added 5
% lines of code in our soundness proof, and effect safety proof remains
% the same.

\subsubsection{What If a Function Has More than One Side Effect}

There is no support for a function with more than one kind of effects
in current system. For example, in the following code snippet, $c1$
cannot be used in the function body, as it's removed by \emph{pure} in
the typing of the function body.

\begin{lstlisting}[language=Scala]
  def error(e:Error)(c1:IO)(c2:Throw) = {
    println("error happen!", c1)  // Error, can't capture c1
    throw c2 e
  }
\end{lstlisting}

It's straight-forward to extend the system with pairs or tuples to
overcome this limitation. However, this is not an issue for later
systems with \emph{free functions}, thus we don't pursue the extension
of pairs and tuples here.

\subsection{Soundness}

\label{sec:stlc-pure-soundness}

We follow the standard formulation of soundness in TAPL
\cite{bpierce2002types}, which consists of \emph{progress} and
\emph{preservation}, defined as follows:

\begin{theorem}[Progress]
If $\varnothing \vdash t : T$, then either $t$ is a value or there is some
$t'$ with $t \longrightarrow t'$.
\end{theorem}

\begin{theorem}[Preservation]
If $\Gamma \vdash t : T$, and $t \longrightarrow t'$, then $\Gamma
\vdash t' : T$.
\end{theorem}

The proof of progress is the same as the proof in standard
STLC. However, there is a significant difference in the proof of
preservation. The classic proof of preservation for STLC (as shown in
TAPL) depends on a substitution lemma, which is formulated as follows:

\begin{lemma}[Subsitution-Classic]
If $\Gamma,\; x:S \vdash t : T$, and $\Gamma \vdash s : S$, then $\Gamma
\vdash [x \mapsto s]t : T$.
\end{lemma}

However, this substitution lemma doesn't hold in current system. For a
counter-example, let's assume that $\Gamma = \{f: E \to B,\; c:E\}$,
then it's obviously that following two typing relations hold:

$\{f: E \to B,\; c:E\},\; x:B \vdash \lambda z:B.\,x \; : \; B \to
  B$ \\
$\{f: E \to B,\; c:E\},\; x:B \vdash f \; c \; : \; B$

However, the following typing relation doesn't hold if we replace $x$
with $f \; c$.

$\{f: E \to B,\; c:E\} \vdash \lambda z:B.\,f \; c \; : \; B \to B$.

In fact, the substituted term $\lambda z:B.\,f \; c$ cannot be typed,
as according to the typing rule \textsc{T-Abs}, it cannot capture the
capability variable $c$ in the environment. To overcome this problem,
we stipulate that the term $s$ must be a value. Remember that in
current system, both lambda abstractions and variables are values,
thus substitution of variables of the capability type E and the base
type B can happen. The new formulation is as follows:

\begin{lemma}[Subsitution-New]
  If $\Gamma,\; x:S \vdash t : T$, s is a value and
  $\Gamma \vdash s : S$, then $\Gamma \vdash [x \mapsto s]t : T$.
\end{lemma}

Interestingly, this strict evaluation requirement contrasts
capability-based effect systems with monad-based effect systems. In
Haskell, if strict evaluation is adopted, it will be impossible to
track effects in the type system, as demonstrated by following code
snippet:

\begin{lstlisting}[language=Haskell]
  inc n = (\x -> n + 1) (putStrLn (show n))
\end{lstlisting}

The function \emph{inc} has the type
$(Num\;a, Show\;a) \Rightarrow a \to a$. By just checking its type, we
would think it has no side effects because no IO monads appear in the
type signature. However, if Haskell adopts strict evaluation, the
function call \emph{putStrLn} will be executed, thus breaking the
monad-based effect system.

\subsection{Effect Safety}

Does the system really work? This question prompts us to formulate and
prove effect safety of the system. We start by formulating effect
safety informally, then put forward a formal formulation, and finally
prove effect safety of the system.

\subsubsection{Informal Formulation}

A straight-forward violation of effect safety is for functions that
are taken as pure to have side effects inside the function body. Thus,
a tentative formulation would be as follows:

\begin{definition}[Effect-Safety-Informally-1]
A function typed in a pure environment cannot have side effects inside.
\end{definition}

However, this formulation is obviously problematic, as we know stoic
functions can have side effects if it takes a capability
parameter. Thus, we need to restrict the functions to those not taking
capability parameters:

\begin{definition}[Effect-Safety-Informally-2]
  A function, not taking any capability parameter and typed in a pure
  environment, cannot have side effects inside.
\end{definition}

This formulation looks more satisfactory, but it's a little
cumbersome. If we inspect the typing rule \textsc{T-Abs} closely, we
can find that if \emph{S} is not a capability type,
$pure(\Gamma),\; x: S$ is equal to $pure(\Gamma,\; x: S)$.

\infrule[T-Abs]
{ pure(\Gamma),\; x: S \vdash t_2 : T }
{ \Gamma \vdash \lambda x:S.\;t_2 : S \to T }

Thus, instead of saying the function $\lambda x:S.t_2$ cannot have
side effects inside, we say the term $t_2$ cannot have side effects in
a pure environment. As we know, capabilities are required to make side
effects. Thus, the term $t_2$ cannot have side effects if we cannot
construct a term of the capability type E in a pure environment. This
observation leads us to the following statement of effect safety:

\begin{definition}[Effect-Safety-Informally-3]
  It's impossible to construct a term of the capability type E in a
  pure environment.
\end{definition}

However, this formulation cannot be proved.  For a counter-example,
let's assume $\Gamma = \{f: B \to E, \; x: B\}$. It's obvious that
$\Gamma$ is pure, but we can construct the term $f \; x$ of the
capability type E.

The cause of the problem is that in a pure environment, there might
exist non-inhabitable types like $B \to E$.  Existence of
non-inhabitable types in a pure environment doesn't pose a problem to
the system; a function taking a parameter of a non-inhabitable type
can never be actually called, thus is always effect-safe. So we only
need to consider environments with only variables of inhabitable
types.

To convince readers that current system is effect-safe, we need to
exclude and only exclude non-inhabitable types from the pure
environment and then prove that it is impossible to construct a term
of the capability type E in this restricted environment. We arrive at
the following formulation:

\begin{definition}[Effect-Safety-Informally-4]
  It's impossible to construct a term of the capability type E in a
  pure environment with only variables of inhabitable types.
\end{definition}

\subsubsection{Inhabitability}

We need to define the concept \emph{inhabitability} precisely. What
types are inhabitable? Obviously, if $\varnothing \vdash t: T$, then T
is inhabitable.  However, given a typing relation
$\Gamma \vdash t: T$, we cannot immediately conclude that T is
inhabitable. We need to ensure that $\Gamma$ only contains inhabitable
types. Otherwise, any type is inhabitable if $\Gamma$ contains a
variable of the corresponding type.

An intuition is that, given $\Gamma \vdash t: T$, $x:S \in \Gamma$ and
S is inhabitable, we can remove x:S from $\Gamma$ and substitute x in
the term t with the term that inhabits the type S to obtain a new term
$t'$. The substitution lemma tells us that the new term $t'$ still has
the type T. Continuing this line of thought, we'll find out that all
inhabitable types can be inhabited in the empty environment. Thus, a
tentative definition is as follows:

\begin{definition}[Inhabitability-First-Try]
  A type T is inhabitable if there exists a term t with $\varnothing
  \vdash t : T$.
\end{definition}

However, this definition is not satisfactory in our case, as in
STLC-Pure there doesn't exist values for the base type B and the
capability type E, except variables, and we do want both types to be
inhabitable. A natural approach is to extend the empty environment
with one variable of the base type and one of the capability type:

\begin{definition}[Inhabitability-Second-Try]
  A type T is inhabitable if there exists a term t with
  $x:B,\; y:E \vdash t : T$.
\end{definition}

This definition indeed gives us all inhabitable types in
STLC-Pure. Types like $E$, $B$, $E \to B$, $E \to E$,
$(B \to E) \to E$, etc., are all inhabitable, while types like
$B \to E$ and $E \to B \to E$ are non-inhabitable. However, we can do
better with the following definition:

\begin{definition}[Inhabitability-Final]
  A type T is inhabitable if there exists a value v with the typing
  $x:B,\; y:E \vdash v : T$.
\end{definition}

What if the term $t$ in the second definition is an application? In
that case, $t$ must be able to take a step until it becomes a value
due to \emph{progress} and \emph{normalization} of the
system\footnote{We didn't prove normalization of STLC-Pure, but the
  proof should be similar to the proof in standard STLC. We only
  proved progress in the empty environment, and the proof can be
  adapted to prove progress under $\{x:B, y:E\}$.}. And the
\emph{preservation} theorem tells us the type remains unchanged during
evaluation. This final definition makes proofs related to
inhabitability simpler. It's useful to give a definition of
inhabitable environments as well:

\begin{definition}[Inhabitable-Environment]
  An environment $\Gamma$ is inhabitable if it only contains variables
  of inhabitable types.
\end{definition}

\subsubsection{Formalization}

With the formal definition of inhabitability, we can formalize effect
safety as follows:

\begin{definition}[Effect-Safety-Inhabitability]
  If $\Gamma$ is a pure and inhabitable environment, then there
  doesn't exist $t$ with $\Gamma \vdash t : E$.
\end{definition}

However, this formulation doesn't give rise to a direct proof. In
fact, this statement is too strong. Some non-inhabitable types, such
as $E \to B \to E$, don't enable us to create a term of the type $E$,
thus it's safe to keep them in the environment. This implies it's
possible to impose a looser restriction on $\Gamma$, as long as all
types that can appear in a pure and inhabitable environment can also
appear in $\Gamma$.

When we examine the problem more closely, we found that through the
lens of the \emph{Curry-Howard isomorphism}, effect safety actually
says that it is impossible to prove the capability type $E$ from a
group of ``good'' premises. Thus, we can classify all types
(propositions) into two groups: in one group $E$ cannot be proved and
in the other group $E$ can be proved. This leads us to a formulation
of \emph{healthy environment}\footnote{Sandro Stucki initially
  suggested the idea of using \emph{caprod} for the definition of
  \emph{pure} environments. I developed it to be a formulation of
  \emph{healthy} environments and used it in the proof of effect
  safety.} given in Figure~\ref{fig:stlc-pure-healthy-definition}.

\begin{figure}[h]
\begin{framed}

% multi-column separator
\setlength{\columnseprule}{0.4pt}
\begin{multicols}{2}

\textbf{Capsafe}

\infax[CS-Base]
{ B \quad \text{capsafe} }

\infrule[CS-Fun1]
{ S \quad caprod }
{ S \to T \quad \text{capsafe} }

\infrule[CS-Fun2]
{ T \quad \text{capsafe} }
{ S \to T \quad \text{capsafe} }

\columnbreak

\textbf{Caprod}

\infax[CP-Eff]
{ E \quad caprod }

\infrule[CP-Fun]
{ S \; \text{capsafe} \andalso T \; caprod }
{ S \to T \quad caprod }

\textbf{Healthy}

\infax[H-Empty]
{ \varnothing \quad caprod }

\infrule[H-Var]
{ G \; healthy \andalso T \; \text{capsafe} }
{ G, \; x:T \quad healthy }


\end{multicols}
\end{framed}

\caption{System STLC-Pure Healthy Environment}
\label{fig:stlc-pure-healthy-definition}
\end{figure}

In the definition, types like $B \to B$, $E \to E$, $E \to B$ and
$(B \to E) \to B$ are considered as \emph{capsafe}, while types like
$B \to E$, $(E \to B) \to E$ are considered as \emph{caprod}. Only
\emph{capsafe} types can appear in a \emph{healthy} environment. To
inspect the formalization in detail, we can ask several questions.

\emph{Are capsafe types inhabitable?} The answer is no. The type
$E \to B \to E$ is non-inhabitable. Allowing this type in the healthy
environment doesn't enable us to construct a term of the capability
type E.

\emph{Are inhabitable types capsafe?} The answer is yes, except the
capability type $E$. As the capability type $E$ cannot appear in the
pure environment, this is not a problem.

\emph{Are caprod types non-inhabitable?} Yes, except the capability
type $E$. As $E$ is also excluded in the pure environment, it's
justified to remove it from the healthy environment.

% Intuitively, the rule \textsc{CP-Fun} can be justified by the fact
% that in a healthy environment it is impossible to construct a function
% which takes a \emph{capsafe} parameter and returns a value of
% \emph{caprod} type. Or from a logical point of view, given a proof
% from which E cannot be proved, we cannot transform it to be a proof
% capable of proving E, together with a group of premises incapable of
% proving E.

Why this formulation of healthy environment is acceptable? In short,
it is because the statement \emph{Effect-Safety-Inhabitability} is
logically implied by the more general statement \emph{Effect-Safety}:

\begin{definition}[Effect-Safety]
  If $\Gamma$ is healthy, then there doesn't exist $t$ with
  $\Gamma \vdash t : E$.
\end{definition}

The logical implication holds because a pure and inhabitable
environment is also a healthy environment. This claim has been
formally proved:

\begin{lemma}[Inhabitable-Capsafe]
  If the type T is inhabitable, then either T is capsafe or $T = E$.
\end{lemma}

\begin{theorem}[Inhabitable-Pure-Healthy]
  If $\Gamma$ is pure and inhabitable, then $\Gamma$ is healthy.
\end{theorem}

% From the theorem above, it's obvious that any property proved for a
% healthy environment also holds for a pure and inhabitable
% environment. Therefore, it suffices to prove the statement
% \emph{Effect-Safety}.

\subsubsection{Proof}

The proof of effect safety depends on following lemmas, most of them
are straight-forward to prove. Effect safety follows immediately from
the lemma \emph{Healthy-Capsafe}.


\begin{lemma}[Capsafe-Not-Caprod]
 If type T is capsafe, then T is not caprod.
\end{lemma}

\begin{lemma}[Capsafe-Or-Caprod]
 For any T, T is either capsafe or caprod.
\end{lemma}

% \begin{lemma}[Healthy-Pure]
%   If environment $\Gamma$ is healthy, then $pure \; \Gamma = \Gamma$.
% \end{lemma}

% Following generalized lemma is essential in the proof of effect safety:

\begin{lemma}[Healthy-Capsafe]
  If $\Gamma$ is healthy and $\Gamma \vdash t : T$, then T is capsafe.
\end{lemma}

\begin{theorem}[Effect-Safety]
  If $\Gamma$ is healthy, then there doesn't exist $t$ with
  $\Gamma \vdash t : E$.
\end{theorem}

\subsubsection{An Intuitive Proof}

There exists an intuitive proof of effect safety without resorting to
healthy environments. The main insight is that the statement
\emph{Effect-Safety-Inhabitability} is logically implied by the
statement \emph{Effect-Safety-Intuitive}:

\begin{definition}[Effect-Safety-Inhabitability]
  If $\Gamma$ is a pure and inhabitable environment, then there
  doesn't exist $t$ with $\Gamma \vdash t : E$.
\end{definition}

\begin{definition}[Effect-Safety-Intuitive]
  There doesn't exist value $v$ with $x:B \vdash v : E$.
\end{definition}

The statement \emph{Effect-Safety-Intuitive} trivially holds, because
the value $v$ can either be a variable or a function, in neither case
can it be typed as the capability type E.

Why does the logical implication hold? In short, if
\emph{Effect-Safety-Inhabitability} doesn't hold, then
\emph{Effect-Safety-Intuitive} doesn't hold either. Thus, the latter
logically implies the former.

For a pure and inhabitable environment
$\Gamma = \{x:T, y:S, \dots, z:U\}$, if there exists t with
$\Gamma \vdash t : E$, then the typing relation still holds by
extending the environment with $b:B$:

\begin{center}
$\{b:B, x:T, y:S, \dots, z:U\} \vdash t: E$
\end{center}

The type $U$ is pure and inhabitable, as $\Gamma$ is pure and
inhabitable. According to the definition of \emph{inhabitability},
there exists a value u with $\{b:B, e:E\} \vdash u: U$. As u is a
value, it can be either a variable or a function. If u is a variable,
it can only be $b$, as $U$ is a closed type. If u is a function, we
have $\{b:B\} \vdash u: U$, as the typing rule \textsc{T-Abs} will
remove the binding $e:E$ from the environment. In both cases, we have
$\{b:B\} \vdash u: U$. Now using the substitution lemma, we have:

\begin{center}
$\{b:B, x:T, y:S, \dots\} \vdash [z \mapsto u]t: E$
\end{center}

Continuing this process, we can reduce the typing environment to be
$\{b:B\}$ and the term to $t'$:

\begin{center}
$\{b:B\} \vdash t': E$
\end{center}

Now combining \emph{progress}\footnote{We only proved that progress
  holds in an empty environment, it's easy to prove it also holds
  under $\{b:B\}$.} and \emph{normalization}\footnote{We didn't prove
  this theorem, but the proof should be similar to the proof in
  standard STLC.}  of STLC-Pure, $t'$ can take finite evaluation steps
to become a value $v$:

\begin{center}
$\{b:B\} \vdash v: E$
\end{center}

To summarize, we have given an intuitive proof of the following
statement:

\begin{center}
  $\neg \; \text{Effect-Safety-Inhabitability} \to \neg \;
  \text{Effect-Safety-Intuitive}$
\end{center}

By the logical law \emph{contraposition}, we have:

\begin{center}
  $ \text{Effect-Safety-Intuitive} \to
  \text{Effect-Safety-Inhabitability} $
\end{center}

As \emph{Effect-Safety-Intuitive} trivially holds,
\emph{Effect-Safety-Inhabitability} follows by \emph{modus ponens}.

Though conceptually simpler, the mechanized proof necessitates the
proof of the normalization theorem, which is more involved than the
proof based on healthy environments. On the other hand, the approach
based on healthy environments works even if normalization doesn't
hold. Therefore, we don't take the intuitive approach in the formal
development.

% extensions and practicality
