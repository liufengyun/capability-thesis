\section{System STLC-Pure}

This chapter describes a variant of the \emph{simply typed
  lambda-calculus} with the extension of capabilities. We call this
system \emph{STLC-Pure}, because in this system all functions must
observe a variable-capturing discipline.

% In later chapters, we'll see the development of impure systems where
% there are both effect-disciplined functions which observe the
% variable-capturing discipline, and ordinary functions which don't
% observe the variable-capturing discipline.

The system STLC-Pure, though conceptually simple, can quite well
demonstrate the main features of capability-based type-and-effect
systems. We'll first introduce the formalization, then discuss
soundness and effect safety. Concepts introduced here will be a
foundation for more complex systems in later chapters.

\subsection{Definitions}

Formally, STLC-Pure is obtained by introducing a capability type and
imposing a variable capability-capturing discipline on lambda
abstractions.  Figure~\ref{fig:stlc-pure-definition} presents the full
definition of STLC-Pure.

The syntax is almost the same as standard STLC, except the addition of
capability type \emph{E} and addition of variables as values. The
evaluation rules are exactly the same, with standard call-by-value
small-step semantics. The typing rule \textsc{T-Abs} is slightly
changed by performing an operation on the environment. The
peculiarities in the formalization are explained below.

\begin{figure}[h]
\begin{framed}

% multi-column separator
\setlength{\columnseprule}{0.4pt}
\begin{multicols}{2}

\textbf{Syntax}

\begin{tabu} to \linewidth {l l l X[r]}
  t   & ::= &                    & terms:               \\
      &     &  x                 & variable             \\
      &     & $\lambda$ x:T.t    & abstraction          \\
      &     & t t                & application          \\
\\
  v   & ::= &                    & values:              \\
      &     & $\lambda$ x:T.t    & abstraction value    \\
      &     & x                  & variable value       \\
\\
  T   & ::= &                    & types:               \\
      &     & B                  & basic type           \\
      &     & E                  & capability type      \\
      &     & T $\to$ T          & type of functions    \\
\end{tabu}

\hfill\\

\textbf{Evaluation} \hfill \framebox[1.2\width][r]{$t \longrightarrow t'$}

\infrule[E-App1]
{ t_1 \longrightarrow t'_1 }
{ t_1 \; t_2 \longrightarrow t'_1 \; t_2 }

\infrule[E-App2]
{ t_2 \longrightarrow t'_2 }
{ v_1 \; t_2 \longrightarrow v_1 \; t'_2 }

\infax[E-AppAbs]
{ (\lambda x:T.t_1) v_2 \longrightarrow [x \mapsto v_2]t_1 }

\columnbreak

\textbf{Typing}  \hfill \framebox[1.2\width][r]{$\Gamma \vdash x : T$}

\infrule[T-Var]
{ x: T \in \Gamma }
{ \Gamma \vdash x : T }

\infrule[T-Abs]
{ \colorbox{shade}{$pure(\Gamma),\; x: S \vdash t_2 : T$} }
{ \colorbox{shade}{$\Gamma \vdash \lambda x:S.t_2 : S \to T$} }

\infrule[T-App]
{ \Gamma \vdash t_1 : S \to T \andalso \Gamma \vdash t_2 : S }
{ \Gamma \vdash t_1 \; t_2 : T }

\textbf{Pure Environment}

\hfill

\begin{center}
\begin{tabular}{l c l}
pure($\varnothing$)             & = &   $\varnothing$ \\
pure($\Gamma$, x: E)            & = &  pure($\Gamma$) \\
pure($\Gamma$, x: T)  & = &  pure($\Gamma$), x: T     \\
\end{tabular}
\end{center}

\end{multicols}
\end{framed}

\caption{System STLC-Pure}
\label{fig:stlc-pure-definition}
\end{figure}

\subsubsection{Variable-Capturing Discipline}

The most important change to standard STLC lies in the following
typing rule:

\infrule[T-Abs]
{ pure(\Gamma), \; x: S \vdash t_2 : T }
{ \Gamma \vdash \lambda x:S.t_2 : S \to T }

This typing rule imposes a \emph{variable-capturing discipline} on
lambda abstractions. This discipline stipulates that only variables
whose type is not a capability type can be captured in a lambda
abstraction.

The discipline is implemented with the helper function \emph{pure},
which removes any variable bindings of the capability type \emph{E}
from the typing environment. It's easy to verify that the function
\emph{pure} satisfies following properties:

\begin{lemma}[Pure-Distributivity]
  pure (E, F) = pure E, pure F
\end{lemma}

\begin{lemma}[Pure-Idempotency]
  pure (pure E) = pure E
\end{lemma}

Initially, we've tried a variable-capturing discipline where no free
variables can be captured, i.e. $pure(\Gamma) = \varnothing$. While
this definition is certainly effect safe, the system is not very
expressive, as we lose the ability to create closures in the system,
which is usually considered to be an essential feature of functional
programming.

\subsubsection{Stoic Functions and Free Functions}

The variable-capturing discipline makes the functions in STLC-Pure
essentially different from functions in standard STLC. In STLC,
functions can capture any variables in scope, while in STLC-Pure
functions can only capture variables whose type is not a capability
type. To differentiate them (which is important as in later systems
both exist), we call the more effect-disciplined functions \emph{stoic
  functions} (or stoics) and the other \emph{free functions}.

Stoic functions are essential in capability-based effect systems. If
functions are allowed to capture capability variables in scope, it
will be impossible to tell whether a function has side effect or not
(and what kind of effect) by just checking its type. Stoic functions
are effect-disciplined in the sense that the only way for stoic
functions to have side effects is to pass a capability as parameter,
thus it can be captured by the type system.

It's important to note that stoic functions are not necessarily pure
functions. Stoic functions can have side effects, and if they do have
side effects they are honest about that in their type signature. For
example, following function \emph{hello} is a stoic function which has
IO effects\footnote{For the sake of readability, we'll use a syntax
  similar to Scala. However, we'll use $\to$ for the type of stoic
  functions, and $\Rightarrow$ for free functions.}.

\begin{lstlisting}[language=Scala]
  def hello(c:IO) = println("hello, world!", c)
\end{lstlisting}

In the following code snippet, we can be sure that the function
\emph{f} is pure, as it doesn't take any capability as parameter. In
the definition of \emph{f}, it's impossible to call functions to
create side effects, as in STLC-Pure all functions that can have side
effects take some capability as parameter.

\begin{lstlisting}[language=Scala]
  def twice(f: Int -> Int)(x: Int) = f (f x)
\end{lstlisting}

\subsubsection{Where do Effects Come From}

Careful readers might have noticed that in current effect system,
there is no formalization of effects. Indeed, that's an intentional
design choice. We assume the existence of primitive functions like
\emph{println} and \emph{readln}, which take capability parameters to
create side effects.

\begin{lstlisting}[language=Scala]
  def println: String -> IO -> ()
  def readln: IO -> String
\end{lstlisting}


% As the primary concern of this study is IO effects,

\subsubsection{Where do Capabilities Come From}

In current system it is impossible to create capabilities. Where do
capabilities come from?  There are two possible answers. The first one
is that all capabilities are from the run-time and passed to the
program through the \emph{main} method.

The second one is that there are no capabilities. Capabilities can be
erased before evaluation in principle, without affecting meaning of
the program. However, in this study we only state this observation
informally and leave the formal proof to future studies.

\subsubsection{Why Treat Variables as Values}

As discussed above, in current system, there is no way to create
capabilities. It means that a function taking a capability parameter
can never be called. A soundness proof for such a system is not very
convincing. Instead, if we treat variables as values, functions that
takes a capability parameter can be called with a capability
variable. This makes the preservation proof more convincing.

Adding variables as values doesn't break soundness or effect safety of
the system. In fact, by adding variables as values, we only added 5
lines of code in our soundness proof, and effect safety proof remains
the same.

\subsubsection{What If a Function Has More than One Side Effects}

The current system doesn't support a function to have more than one
kind of effects. For example, in the following code snippet, $c1$
can't be used in the function body, as it's removed by \emph{pure} in
the typing of the function body.

\begin{lstlisting}[language=Scala]
  def error(e:Error)(c1:IO)(c2:Throw) = {
    println("error happen!", c1)
    throw c2 e
  }
\end{lstlisting}

It's straight-forward to extend the system with pairs or tuples to
overcome the limit of the current system. However, in later chapters
we'll see systems where there exists functions that are not restricted
by the variable-capturing discipline.  In those systems, this problem
is no longer an issue, thus we don't pursue the extension of pairs and
tuples here.

\subsection{Soundness}

We follow the standard formulation of soundness in TAPL
\cite{bpierce2002types}, which consists of \emph{progress} and
\emph{preservation}, defined as follows:

\begin{theorem}[Progress]
If $\varnothing \vdash t : T$, then either $t$ is a value or there is some
$t'$ with $t \longrightarrow t'$.
\end{theorem}

\begin{theorem}[Preservation]
If $\Gamma \vdash t : T$, and $t \longrightarrow t'$, then $\Gamma
\vdash t' : T$.
\end{theorem}

We proved both theorems in Coq based on the locally-nameless
representation. The proof of progress is the same as in STLC. However,
there is a significant difference in the proof of preservation. In the
classic proof of preservation for STLC (as shown in TAPL), it depends
on a substitution lemma, which is formulated as follows:

\begin{lemma}[Subsitution-Classic]
If $\Gamma,\; x:S \vdash t : T$, and $\Gamma \vdash s : S$, then $\Gamma
\vdash [x \mapsto s]t : T$.
\end{lemma}

However, this substitution lemma doesn't hold in current system. For a
counter-example, let's assume that $\Gamma = \{f: E \to B,\;
  c:E\}$, then it's obviously that following two propositions hold:


$\{f: E \to B,\; c:E\},\; x:B \vdash \lambda z:B.\,x \; : \; B \to
  B$ \\
$\{f: E \to B,\; c:E\},\; x:B \vdash f \; c \; : \; B$

However, following typing relationship doesn't hold if we replace
$x$ with $f \; c$.

$\{f: E \to B,\; c:E\} \vdash \lambda z:B.\,f \; c \; : \; B \to B$.

In fact, the substituted term $\lambda z:B.\,f \; c$ cannot be typed,
as according to the typing rule \textsc{T-Abs}, it cannot capture the
capability variable $c$ in the environment. To overcome this problem,
we stipulate that the term $s$ to be substituted must be a
value. Remember that in current system, both lambda abstractions and
variables are values, thus substitution of capability variables can
also happen. The new formulation is as follows:

\begin{lemma}[Subsitution-New]
  If $\Gamma,\; x:S \vdash t : T$, s is a value and
  $\Gamma \vdash s : S$, then $\Gamma \vdash [x \mapsto s]t : T$.
\end{lemma}

Interestingly, this strict evaluation requirement contrasts
capability-based effect systems with monad-based effect systems. In
Haskell, if it adopts strict evaluation, it will be impossible to
track effects in the type system, as demonstrated by following code
snippet:

\begin{lstlisting}[language=Haskell]
  inc n = (\x -> n + 1) (putStrLn (show n))
\end{lstlisting}

The function \emph{inc} has the type
$(Num\;a, Show\;a) \Rightarrow a \to a$. By just checking its type, we
would think it has no side effects because no IO monads appear in the
type signature. However, if Haskell adopts strict evaluation, the
function call \emph{putStrLn} will be executed, thus breaks the
monad-based effect system.

\subsection{Effect Safety}

One question that can be asked for any type-and-effect system is that,
does the system really work? This question prompts us to formulate and
prove effect safety of the system.

\subsubsection{Informal Formulation}

We start by formulating the effect safety informally. A tentative
formulation would be as follows:

\begin{definition}[Effect-Safety-Informally-1]
A function typed in pure environment cannot have side effects inside.
\end{definition}

However, this formulation is obviously problematic, as we know stoic
functions can have side effects if it takes a capability
parameter. Thus, we need to restrict the functions to those not taking
capability parameters:

\begin{definition}[Effect-Safety-Informally-2]
  A function (without capability parameter) typed in pure environment
  cannot have side effects inside.
\end{definition}

This formulation looks more satisfactory, but it's a little
cumbersome, let's transform it into a more succinct statement. If we
inspect the T-Abs typing rule closely, we can find that if \emph{S} is
not a capability type, $pure(\Gamma),\; x: S$ is equal to
$pure(\Gamma,\; x: S)$.

\infrule[T-Abs]
{ pure(\Gamma),\; x: S \vdash t_2 : T }
{ \Gamma \vdash \lambda x:S.\;t_2 : S \to T }

Thus, instead of saying a function $\lambda x:S.t_2 : S$ cannot have
side effects inside, we can say the term $t_2$ cannot have side
effects in pure environment. As we know, to have side effects, we must
supply a capability as parameter. Thus, the term $t_2$ cannot have
side effects if we cannot construct a term of capability type in pure
environment. This observation leads us to the following statement of
effect safety:

\begin{definition}[Effect-Safety-Informally-3]
 It's impossible to construct a term $t$ of capability type in pure environment.
\end{definition}

This formulation looks very promising, but in fact it cannot be proved!
For a counter-example, let's assume $\Gamma = \{f: B \to E, \; x:
B\}$. It's obvious that $\Gamma$ is pure, but we can construct the
term $f \; x$ which is of the capability type.

The cause of the problem is that in pure environment, there could exist
ill types like $B \to E$, which is non-inhabitable in ``healthy''
environments.  Existence of these ill types in pure environment
doesn't pose a problem to the system, because if a function takes a
non-inhabitable parameter, the function can never be actually called.

To convince the readers that the current system is effect-safe, we
need to carefully exclude and only exclude ill types in the pure
environment and then prove that it is impossible to construct a term
of capability type in this restricted environment. Thus we arrive at
following formulation:

\begin{definition}[Effect-Safety-Informally-4]
  It's impossible to construct a term $t$ of capability type in any
  pure environment without variables of non-inhabitable types.
\end{definition}

\subsubsection{Formal Formulation}

However, the insight in the informal statement doesn't give rise to a
direct proof of effect safety. We need a formulation not based on
inhabitability of types. When we examine the problem more closely, we
found that through the lens of \emph{Curry Howard Isomorphism}, the
effect safety actually says that it is impossible to prove the
capability type $E$ from a group of ``good'' premises. Then, we can
classify all types (propositions) into two groups: in one group $E$
cannot be proved and in the other group $E$ can be proved. This leads
us to a formulation of \emph{healthy environment} given in
Figure~\ref{fig:stlc-pure-healthy-definition}. Given the formulation
of \emph{healthy environment}, the effect safety can be formally
stated:

\begin{definition}[Effect-Safety]
  If $\Gamma$ is healhty, then there doesn't exist $t$ with
  $\Gamma \vdash t : E$.
\end{definition}

\begin{figure}[h]
\begin{framed}

% multi-column separator
\setlength{\columnseprule}{0.4pt}
\begin{multicols}{2}

\textbf{Capsafe}

\infax[CS-Base]
{ B \quad \text{capsafe} }

\infrule[CS-Fun1]
{ S \quad caprod }
{ S \to T \quad \text{capsafe} }

\infrule[CS-Fun2]
{ T \quad \text{capsafe} }
{ S \to T \quad \text{capsafe} }

\columnbreak

\textbf{Caprod}

\infax[CP-Eff]
{ E \quad caprod }

\infrule[CP-Fun]
{ S \; \text{capsafe} \andalso T \; caprod }
{ S \to T \quad caprod }

\textbf{Healthy}

\infax[H-Empty]
{ \varnothing \quad caprod }

\infrule[H-Var]
{ G \; healthy \andalso T \; \text{capsafe} }
{ G, \; x:T \quad healthy }


\end{multicols}
\end{framed}

\caption{System STLC-Pure Healthy Environment}
\label{fig:stlc-pure-healthy-definition}
\end{figure}

In the definition, types like $B \to B$, $E \to E$, $E \to B$ and
$(B \to E) \to B$ are considered as capsafe, while types like
$B \to E$, $(E \to B) \to E$ are considered as caprod. To inspect the
formalization in detail, we can ask several questions.

\emph{Are capsafe types inhabitable?} The answer is no. The type
$E \to B \to E$ is non-inhabitable. However, allowing this type in
healthy environment doesn't enable us to construct a term of
capability type.

\emph{Are caprod types non-inhabitable?} Yes, except the capability
type $E$. As $E$ is also excluded in pure environment, thus it's
justified to remove it in the healthy environment. The CP-Fun rule is
justified by the fact that in a healthy environment it is impossible
to construct a function which takes a capsafe parameter and returns a
value of caprod type. Or from a logical point of view, given a proof
from which E can't be proved, we can't transform it to be a proof
capable of proving E, together with a group of premises incapable of
proving E.

\emph{Why this formulation of healthy environment is acceptable?} This
formulation is acceptable because it only rejects non-inhabitable
types in pure environments, which is the key insight in the informal
formulation of effect safety. It's incorrect to reject inhabitable
types, but it's acceptable to keep non-inhabitable types, as long as
effect safety can be proved.

Note that it's possible to formalize inhabitability and prove
rigorously that all caprod types are actually non-inhabitable. We
leave it to future studies and only resort to informal arguments for
the correctness of the formulation of \emph{healthy environment} here.

\subsubsection{Proof}

The proof of effect safety depends on following lemmas, most of them
are straight-forward to prove. Effect safety follows immediately from
the lemma \emph{Healthy-Capsafe}.


\begin{lemma}[Capsafe-Not-Caprod]
 If type T is capsafe, then T is not caprod.
\end{lemma}

\begin{lemma}[Capsafe-Or-Caprod]
 For any T, T is either capsafe or caprod.
\end{lemma}

% \begin{lemma}[Healthy-Pure]
%   If environment $\Gamma$ is healthy, then $pure \; \Gamma = \Gamma$.
% \end{lemma}

% Following generalized lemma is essential in the proof of effect safety:

\begin{lemma}[Healthy-Capsafe]
  If $\Gamma$ is healthy and $\Gamma \vdash t : T$, then T is capsafe.
\end{lemma}

\begin{theorem}[Effect-Safety]
  If $\Gamma$ is healhty, then there doesn't exist $t$ with
  $\Gamma \vdash t : E$.
\end{theorem}

% \subsection{Discussion}

% extensions and practicality
