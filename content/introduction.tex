\section{Introduction}

The objective of this study is to explore the theoretical foundations
as well as conceptual possibilities of capability-based effect
systems. In this chapter, we'll discuss the motivation, the core ideas
and contributions of this study.

\subsection{Motivation}

The main motivation is to turn Scala into an effect-disciplined
programming language. Currently, Scala doesn't have the ability to
track effects in the type system. This poses a problem in distributed
and parallel programming. For example, in parallel computing, there's
often the need to stipulate that the functions passed to \emph{pmap}
have no side effects.

\begin{lstlisting}[language=Scala]
def pmap(xs: List[Int], f: Int => Int): List[Int]
\end{lstlisting}

However, currently it's impossible for the library author to impose
the constraint in the type system. A type-and-effect system has been
proposed for Scala\cite{lukas2014effect}, but is not well received in
the Scala community, mainly due to its notational verbosity in
handling \emph{effect polymorphism}. The problem of \emph{effect
  polymorphism} can be illustrated by the \emph{map} function:

\begin{lstlisting}[language=Scala]
  def map[A,B](f: A => B)(l: List[A]) = l match {
    case Nil => Nil
    case x::xs => f(x)::map(f)(l)
  }
\end{lstlisting}

In an effect system, the effect of \emph{map} depends on the passed in
function \emph{f}. If \emph{f} has IO effects, then \emph{map} also
has IO effects. If \emph{f} is pure, then \emph{map} is pure as
well. In order to make effect polymorphism work, proposed
type-and-effect systems usually require programmers to annotate the
function \emph{map} to say that its effect depends on the function
\emph{f}, which incurs some notational burden on programmers.

Monad-based effect systems have a success in Haskell. However, as we
will see, monad-based effect systems only work in lazy evaluation
languages, which is not a good fit for Scala because Scala is
strict. On the other hand, monad-based effect systems also face the
problem of \emph{effect polymorphism}. As reported in Chapter 1
section 6 of the thesis\cite{lippmeier2009type}, almost every general
purpose higher-order function in Haskell needs both a monadic version
and non-monadic version. For example, following code snippet shows the
signature for two versions of the \emph{map} function. The duplication
of code is a pity for programmers.

\begin{lstlisting}[language=Haskell]
  map :: (a -> b) -> List a -> List b
  mapIO :: (a -> IO b) -> List a -> IO (List b)
\end{lstlisting}

Given this situation, we are compelled to find a new approach to the
effect polymorphism problem, and \emph{capability-based} effect
systems seem to be a promising direction.

\subsection{Capability-based Effect Systems}

The central idea of \emph{capability-based} effect system is that an
instance of capability is required in order to make side effects. If
capabilities are passed as function parameters, by tracking
capabilities in the type system we can track effects in the program.

To ensure that capabilities are passed through function parameters,
instead of being captured from the environment, we need to impose a
\emph{variable-capturing discipline}, stipulating that capability
variables cannot be captured. Functions observe the discipline are
called \emph{stoic functions}, while functions don't oberve the
discipline are called \emph{free functions}.

With the combination of free functions and stoic functions,
\emph{capability-based} effect systems can solve the problem of
\emph{effect polymorphism} easily, while incurring no syntactical
burden. For example, instead of having two versions of \emph{map},
following \emph{map} is effect polymorphic.

\begin{lstlisting}[language=Scala]
  def map[A,B](f: A => B)(l: List[A]) = l match {
    case Nil => Nil
    case x::xs => f(x)::map(f)(l)
  }
\end{lstlisting}

When the passed function \emph{f} is pure, \emph{map} is pure as
well. In contrast, if \emph{f} is impure, then \emph{map} is also
impure. No additional annotation is required for effect polymorphism
to work. We'll see details of effect polymorphism in system
STLC-Impure.

\subsection{Contributions}

The main contributions of this study are as follows:

\begin{itemize}
\item Formulated and proved soundness and effect safety of four
  capability-based effect systems, which can serve as the fundation
  for implementing capability-based effect system in functional
  programming languages. The formalization is done in Coq based on the
  locally-nameless representation\cite{chargueraud-11-ln} and hosted
  on github\footnote{\url{https://github.com/liufengyun/stoic}}.
\item Proposed an approach to solve the problem of \emph{effect
    polymorphism} in capability-based effect systems with both free
  functions and stoic functions. The solution is much simpler and more
  elegant than in type-and-effect systems and monad-based effect
  systems.
\end{itemize}

\subsection{Structure of Report}

In following chapters, we'll introduce four systems of increasing
complexity, namely STLC-Pure, F-Pure, STLC-Impure and F-Impure.

\begin{itemize}
\item STLC-Pure is an variant of STLC with only stoic functions.
\item STLC-Impure is an extension of STLC-Pure with free functions and subtyping.
\item F-Pure is an extension of STLC-Pure with universal types.
\item F-Impure is an extension of F-Pure with free functions.
\end{itemize}
