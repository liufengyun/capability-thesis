\section{System F-Impure}

In this chapter, we present the system \emph{F-Impure}, which is an
extension of the system F-Pure with free functions. It can also be
seen as an extension of the system STLC-Impure with universal types,
but without subtyping. Extending the system with subtyping would lead
us to bounded quantification, which we are still working on. Given the
importance of parametric polymorphism and the fact that subtyping is
not a necessary add-on of functional programming, the system
\emph{F-Impure} deserves a separate presentation here.

We'll first introduce the formalization, then discuss soundness and
effect safety. In the discussion, we'll focus on its difference from
the system STLC-Impure and F-Pure.

\begin{figure}
\begin{framed}

% multi-column separator
\setlength{\columnseprule}{0.4pt}
\begin{multicols}{2}

\textbf{Syntax}

\begin{tabu} to \linewidth {l l l X[r]}
  t   & ::= &                                      & terms:               \\
      &     &  x                                   & variable             \\
      &     & $\lambda$ x:T.t                      & abstraction          \\
      &     & t t                                  & application          \\
      &     & $\lambda$ X.t                        & type abstraction     \\
      &     & t [T]                                & type application     \\
\\
  v   & ::= &                    & values:              \\
      &     & $\lambda$ x:T.t    & abstraction value    \\
      &     & x                  & variable value       \\
      &     & $\lambda X.t$      & type abstraction value  \\
\\
  T   & ::= &                       & types:               \\
      &     & X                     & type variable        \\
      &     & B                     & basic type           \\
      &     & E                     & capability type      \\
      &     & T $\to$ T             & type of stoic funs   \\
      &     & \colorbox{shade}{T $\Rightarrow$ T}     & type of free funs    \\
      &     & $\forall$ X.T         & universal type       \\
\end{tabu}

\hfill\\

\textbf{Evaluation} \hfill \framebox[1.2\width][r]{$t \longrightarrow t'$}

\infrule[E-App1]
{ t_1 \longrightarrow t'_1 }
{ t_1 \; t_2 \longrightarrow t'_1 \; t_2 }

\infrule[E-App2]
{ t_2 \longrightarrow t'_2 }
{ v_1 \; t_2 \longrightarrow v_1 \; t'_2 }

\infax[E-AppAbs]
{ (\lambda x:T.t_1) v_2 \longrightarrow [x \mapsto v_2]t_1 }

\infrule[E-TApp]
{ t_1 \longrightarrow t'_1 }
{ t_1 \; [T_2] \longrightarrow t'_1 \; [T_2] }

\infax[E-TappTabs]
{ (\lambda X.t_1) [T_2] \longrightarrow [X \mapsto T_2]t_1 }

\columnbreak

\textbf{Typing}  \hfill \framebox[1.2\width][r]{$\Gamma \vdash x : T$}

\infrule[T-Var]
{ x: T \in \Gamma }
{ \Gamma \vdash x : T }

\infrule[T-Abs2]
{ pure(\Gamma),\; x: S \vdash t_2 : T }
{ \Gamma \vdash \lambda x:S.t_2 : S \to T }

\infrule[T-Abs2]
{  \colorbox{shade}{$\Gamma,\; x: S \vdash t_2 : T$} }
{  \colorbox{shade}{$\Gamma \vdash \lambda x:S.t_2 : S \Rightarrow T$} }

\infrule[T-Degen]
{ \colorbox{shade}{$\Gamma \vdash t : S \to T$} }
{ \colorbox{shade}{$\Gamma \vdash t : S \Rightarrow T$} }

\infrule[T-App]
{ \Gamma \vdash t_1 : S \to T \andalso \Gamma \vdash t_2 : S }
{ \Gamma \vdash t_1 \; t_2 : T }

\infrule[T-TAbs]
{ pure(\Gamma),\; X \vdash t_2 : T }
{ \Gamma \vdash \lambda X.t_2 : \forall X. T }

\infrule[T-TApp]
{ \Gamma \vdash t_1 : \forall X.T \andalso T_2 \neq E }
{ \Gamma \vdash t_1 \; [T_2] : [X \mapsto T_2]T }

\textbf{Pure Environment}

\hfill

\begin{center}
\begin{tabular}{l c l}
pure($\varnothing$)             & = &   $\varnothing$ \\
pure($\Gamma$, x: E)            & = &  pure($\Gamma$) \\
\rowcolor{gray!40}
pure($\Gamma$, x: S $\Rightarrow$ T)  & = &  pure($\Gamma$) \\
pure($\Gamma$, x: T)  & = &  pure($\Gamma$), x: T     \\
pure($\Gamma$, X)  & = &  pure($\Gamma$), X  \\
\end{tabular}
\end{center}

\hfill\\

\end{multicols}
\end{framed}

\caption{System F-Impure}
\label{fig:f-impure-definition}
\end{figure}

\subsection{Definitions}

Figure~\ref{fig:f-impure-definition} presents the full definition of
F-Pure, with the difference from the system F-Impure highlighted. As
can be seen from the figure, we introduced free function types and
added a typing rule for free functions. As we have no subtyping in the
system, we have to add a \textsc{T-Degen} rule to restore the
subtyping relation between stoic function types and free function
types.

As in STLC-Impure, we adapted the definition of \emph{pure} to exclude
free function types from pure environments. If stoic functions have
access to free functions, we'll loose the ability to track the effects
of stoic functions in the type system.

A different design choice we could make is to introduce free universal
abstractions in the system. Such an extension will not be very useful
in real-world programming, as in practice polymorphic functions rarely
capture free variables, not mention capability variables. Thus for the
sake of simplicity, we don't pursue the extension.

\subsection{Soundness}

We proved both progress and preservation of the system. The subtleties
in the proof are the same as stated in the system F-Pure.

\begin{theorem}[Progress]
If $\varnothing \vdash t : T$, then either $t$ is a value or there is some
$t'$ with $t \longrightarrow t'$.
\end{theorem}

\begin{theorem}[Preservation]
If $\Gamma \vdash t : T$, and $t \longrightarrow t'$, then $\Gamma
\vdash t' : T$.
\end{theorem}

\subsection{Effect Safety}

We first introduce the formulation, which is a combination of the
formulation in STLC-Impure and F-Pure, then discuss the proof of
effect safety.

\subsubsection{Formulation}

As in the system STLC-Impure, in the presence of free functions, we
need two statements of effect safety:

\begin{definition}[Effect-Safety-Inhabitability-1]
  If $\Gamma$ is a pure and inhabitable environment, then there
  doesn't exist $t$ with $\Gamma \vdash t : E$.
\end{definition}

\begin{definition}[Effect-Safety-Inhabitability-2]
  If $\Gamma$ is a pure and inhabitable environment, and
  $\Gamma \vdash t_1 \; t_2 : T$, then there exists U, V such that
  $\Gamma \vdash t_1 : U \to V$.
\end{definition}


As in the system STLC-Impure, the proof of these two statements
depends on two weaker and more general statements about \emph{healthy
  environments}. Given that we've seen how universal types and free
function types are extended in the formulation of \emph{healthy
  environment}, we can easily combine them to arrive at the
formulation shown in Figure~\ref{fig:f-impure-healthy-definition},
with the changes from F-Pure highlighted.

\begin{figure}[h]
\begin{framed}

% multi-column separator
\setlength{\columnseprule}{0.4pt}
\begin{multicols}{2}

\textbf{Capsafe}

\infax[CS-Base]
{ B \quad \text{capsafe} }

\infrule[CS-Fun1]
{ S \quad caprod }
{ S \to T \quad \text{capsafe} }

\infrule[CS-Fun2]
{ T \quad \text{capsafe} }
{ S \to T \quad \text{capsafe} }

\infrule[CS-Fun3]
{ \colorbox{shade}{$S \quad caprod$} }
{ \colorbox{shade}{$S \Rightarrow T \quad \text{capsafe}$} }

\infrule[CS-Fun4]
{ \colorbox{shade}{$T \quad \text{capsafe}$} }
{ \colorbox{shade}{$S \Rightarrow T \quad \text{capsafe}$} }

\infrule[CS-All]
{ [X \mapsto B]T \; \text{capsafe} \andalso [X \mapsto E]T \; \text{capsafe} }
{ \forall X.T \quad \text{capsafe} }

\columnbreak

\textbf{Caprod}

\infax[CP-Eff]
{ E \quad caprod }

\infrule[CP-Fun1]
{ S \; \text{capsafe} \andalso T \; caprod }
{ S \to T \quad caprod }

\infrule[CP-Fun2]
{ \colorbox{shade}{$S \; \text{capsafe} \andalso T \; caprod$} }
{ \colorbox{shade}{$S \Rightarrow T \quad caprod$} }

\infrule[CP-All]
{ [X \mapsto B]T \; caprod \quad or \quad [X \mapsto E]T \; caprod }
{ \forall X.T \quad caprod }

\textbf{Healthy}

\infax[H-Empty]
{ \varnothing \quad caprod }

\infrule[H-Var]
{ G \; healthy \andalso T \; \text{capsafe} }
{ G, \; x:T \quad healthy }

\infrule[H-TVar]
{ G \quad healthy }
{ G, \; X \quad healthy }

\hfill\\

\end{multicols}
\end{framed}

\caption{System F-Impure Healthy Environment}
\label{fig:f-impure-healthy-definition}
\end{figure}

Why this formulation of healthy environment is acceptable? In short,
it's because the statement \emph{Effect-Safety-Inhabitability-1} and
\emph{Effect-Safety-Inhabitability-2} are logically implied by the
statement \emph{Effect-Safety-1} and \emph{Effect-Safety-2}
respectively.

\begin{definition}[Effect-Safety-1]
  If $\Gamma$ is healthy, there doesn't exist $t$ with
  $\Gamma \vdash t : E$.
\end{definition}

\begin{definition}[Effect-Safety-2]
  If $\Gamma$ is pure and healthy, and $\Gamma \vdash t_1 \; t_2 : T$,
  then there exists U, V such that $\Gamma \vdash t_1 : U \to V$.
\end{definition}

The logical implications hold because a pure and inhabitable
environment is also a healthy (and pure) environment. This claim has
been formally proved:

\begin{lemma}[Inhabitable-Capsafe]
  If the type T is inhabitable, then either T is capsafe or $T = E$ or
  T is a free function type.
\end{lemma}

\begin{theorem}[Inhabitable-Pure-Healthy]
  If $\Gamma$ is pure and inhabitable, then $\Gamma$ is also healthy.
\end{theorem}

\begin{theorem}[Inhabitable-Pure-Healthy']
  If $\Gamma$ is pure and inhabitable, then $\Gamma$ is pure and
  healthy.
\end{theorem}

Note that the last theorem follows immediately from the second one, as
we already know from the premise that $\Gamma$ is pure.

% From the theorem above, it's obvious that any property proved for a
% healthy environment also holds for a pure and inhabitable
% environment. Thus, it suffices to prove the following statements about
% healthy environments.

\subsubsection{Proof}

The proof of the first effect safety statement is almost the same as
in the system F-Pure, thus we omit here.

The proof of the second statement of effect safety faces the same
problem as in the system STLC-Impure. We need to assume a set of
axioms, as shown in Figure~\ref{fig:f-impure-axioms}, with the newly
added axioms highlighted. The justification for the axiom
\textsc{Ax-All} and \textsc{Ax-Var} is the same as the justification
for the axiom \textsc{Ax-Base} in STLC-Impure. In short, because the
outer function is stoic and the first parameter is pure, the inner
function cannot capture variables of capabilities or free functions,
thus it's fair enough to type the inner function as stoic.

\begin{figure}[h]
\begin{framed}

% multi-column separator
% \setlength{\columnseprule}{0.4pt}
\begin{multicols}{2}

\infrule[Ax-Base]
{ \Gamma \vdash t : B \to S \Rightarrow T }
{ \Gamma \vdash t : B \to S \to T }

\hfill\\

\infrule[Ax-Var]
{ \colorbox{shade}{$\Gamma \vdash t : X \to S \Rightarrow T$} }
{ \colorbox{shade}{$\Gamma \vdash t : X \to S \to T$} }

\hfill\\

\infrule[Ax-Poly]
{ \Gamma \vdash t_2 : U \to V \\
  \Gamma \vdash t_1 : (U \Rightarrow V) \to S \Rightarrow T }
{ \Gamma \vdash t_1 \; t_2 : S \to T }

\columnbreak

\infrule[Ax-All]
{ \colorbox{shade}{$\Gamma \vdash t : \forall X.T \to S \Rightarrow T$} }
{ \colorbox{shade}{$\Gamma \vdash t : \forall X.T \to S \to T$} }

\hfill\\

\infrule[Ax-TApp]
{ \colorbox{shade}{$\Gamma \vdash t : \forall X.T_1 \Rightarrow T_2$} }
{ \colorbox{shade}{$\Gamma \vdash t \; [U] : [X \mapsto U]T_1 \to [X
    \mapsto U]T_2$} }

\hfill\\

\infrule[Ax-Stoic]
{ \Gamma \vdash t : (U \to V) \to S \Rightarrow T }
{ \Gamma \vdash t : (U \to V) \to S \to T }

\end{multicols}
\end{framed}

\caption{System F-Impure Axioms}
\label{fig:f-impure-axioms}
\end{figure}

The justification for the axiom \textsc{Ax-TApp} is similar. If a term
$t$ can be typed as $\forall X.T_1 \Rightarrow T_2$ under $\Gamma$,
according to the typing rule \textsc{T-All}, the whole term can be
typed under $pure(\Gamma)$. Then the inner lambda abstraction can be
typed under $pure(\Gamma), X$, which is equal to $pure(\Gamma,
X)$. Thus, it's fair enough to type the inner function as stoic.

The justification for the axiom \textsc{Ax-Poly} is the same as given
in STLC-Impure, thus we omit here.

Assuming these axioms, it's straight-forward to prove a lemma
\emph{Healthy-Pure-Stoic}, and the second statement of effect safety
follows immediately from the lemma.

\begin{lemma}[Healthy-Pure-Stoic]
  If $\Gamma$ is pure and healthy,  and $\Gamma \vdash t : S
  \Rightarrow T$, then $\Gamma \vdash t : S \to T$.
\end{lemma}

\begin{theorem}[Effect-Safety-2]
  If $\Gamma$ is pure and healthy, and $\Gamma \vdash t_1 \; t_2 : T$,
  then there exists U, V such that $\Gamma \vdash t_1 : U \to V$.
\end{theorem}
