\section{System F-Pure}

In this chapter, we present the system F-Pure, which is an extension
of the system STLC-Pure with universal types. In this system, not only
functions need to observe a variable-capturing discipline, type
abstractions also need to observe the same variable-capturing
discipline.

We'll first introduce the formalization, then discuss soundness and
effect safety. In the discussion, we'll focus on its difference from
the system STLC-Pure.

\begin{figure}
\begin{framed}

% multi-column separator
\setlength{\columnseprule}{0.4pt}
\begin{multicols}{2}

\textbf{Syntax}

\begin{tabu} to \linewidth {l l l X[r]}
  $t$ & ::= &                                      & terms:               \\
      &     & $x$                                  & variable             \\
      &     & $\uplambda x{:}T.\, t$               & abstraction          \\
      &     & $t t$                                & application          \\
      &     & \colorbox{shade}{$\uplambda X.\, t$} & type abstraction     \\
      &     & \colorbox{shade}{$t [T]$}            & type application     \\
\\
  $v$ & ::= &                    & values:              \\
      &     & $\uplambda x{:}T.\, t$ & abstraction value    \\
      &     & $x$                & variable value       \\
      &     & \colorbox{shade}{$\uplambda X.\, t$}    & type abstraction value  \\
\\
  $T$ & ::= &                         & types:               \\
      &     & \colorbox{shade}{$X$}   & type variable        \\
      &     & $B$                     & basic type           \\
      &     & $E$                     & capability type      \\
      &     & $T \to T$              & type of functions    \\
      &     & \colorbox{shade}{$\forall X. T$} & universal type       \\
\end{tabu}

\hfill\\

\textbf{Evaluation} \hfill \framebox[1.2\width][r]{$t \longrightarrow t'$}

\infrule[E-App1]
{ t_1 \longrightarrow t'_1 }
{ t_1 \; t_2 \longrightarrow t'_1 \; t_2 }

\infrule[E-App2]
{ t_2 \longrightarrow t'_2 }
{ v_1 \; t_2 \longrightarrow v_1 \; t'_2 }

\infax[E-AppAbs]
{ (\uplambda x{:}T.\, t_1) v_2 \longrightarrow [x \mapsto v_2]t_1 }

\infrule[E-TApp]
{ \colorbox{shade}{$t_1 \longrightarrow t'_1$} }
{ \colorbox{shade}{$t_1 \; [T_2] \longrightarrow t'_1 \; [T_2]$} }

\infax[E-TappTabs]
{ \colorbox{shade}{$(\uplambda X.\, t_1) [T_2] \longrightarrow [X \mapsto T_2]t_1$} }

\columnbreak

\textbf{Typing}  \hfill \framebox[1.2\width][r]{$\Gamma \vdash x : T$}

\infrule[T-Var]
{ x: T \in \Gamma }
{ \Gamma \vdash x : T }

\infrule[T-Abs]
{ pure(\Gamma),\; x: S \vdash t_2 : T }
{ \Gamma \vdash \uplambda x{:}S.\, t_2 : S \to T }

\infrule[T-App]
{ \Gamma \vdash t_1 : S \to T \andalso \Gamma \vdash t_2 : S }
{ \Gamma \vdash t_1 \; t_2 : T }

\infrule[T-TAbs]
{ \colorbox{shade}{$pure(\Gamma),\; X \vdash t_2 : T$} }
{ \colorbox{shade}{$\Gamma \vdash \uplambda X.\, t_2 : \forall X. T$} }

\infrule[T-TApp]
{ \colorbox{shade}{$\Gamma \vdash t_1 : \forall X.T \andalso T_2 \neq E$} }
{ \colorbox{shade}{$\Gamma \vdash t_1 \; [T_2] : [X \mapsto T_2]T$} }

\hfill\\

\textbf{Pure Environment}

\hfill

\begin{center}
\begin{tabular}{l c l}
$pure(\varnothing)$           & = &  $\varnothing$ \\
$pure(\Gamma, \; x: E)$       & = &  $pure(\Gamma)$ \\
$pure(\Gamma, \; x: T)$       & = &  $pure(\Gamma), \; x: T$     \\
\rowcolor{gray!40}
$pure(\Gamma, \; X)$          & = &  $pure(\Gamma), \; X$  \\
\end{tabular}
\end{center}


\end{multicols}
\end{framed}

\caption{System F-Pure}
\label{fig:f-pure-definition}
\end{figure}

\subsection{Definitions}

The system F-Pure extends STLC-Pure with universal
types. Figure~\ref{fig:f-pure-definition} presents the full definition
of F-Pure, with the difference from the system STLC-Pure highlighted.

The extension of syntax and evaluation rules are exactly the same as
the extension of standard STLC with universal types.  The essential
difference lies in the two new typing rules \textsc{T-TAbs} and
\textsc{T-TApp}. The typing rule \textsc{T-TAbs} stipulates that type
abstraction must observe the variable-capturing discipline.

\infrule[T-TAbs]
{ pure(\Gamma),\; X \vdash t_2 : T }
{ \Gamma \vdash \uplambda X.\, t_2 : \forall X. T }

We made this design choice in order to allow universal types to be
present in pure environments. Otherwise, if type abstractions can
capture capability variables, application of a type abstraction could
generate a term of the capability type or have side effects. This
makes it incorrect to have universal types in pure environments, thus
renders universal types useless in the system.

The typing rule \textsc{T-TApp} requires that the type parameter
cannot be the capability type $E$. However, it's allowed to supply
uninhabited types like $B \to E$ as parameter to type abstraction.

% This restriction implies in system F-Pure polymorphism doesn't cover
% capability types.

\infrule[T-TApp]
{ \Gamma \vdash t_1 : \forall X.T \andalso T_2 \neq E }
{ \Gamma \vdash t_1 \; [T_2] : [X \mapsto T_2]T }

Without the restriction, preservation of the system breaks, as can be
seen from the following term $t$, which has the type
$\forall X. X \to B \to X$:

\begin{center}
  $t = \uplambda X. \, \uplambda x{:}X. \, \uplambda y{:}B. \, x$
\end{center}

If we allow $E$ as parameter to type application, the term $t \; [E]$
would have the type $E \to B \to E$. However, after one evaluation
step, the term $\uplambda x{:}E. \, \uplambda y{:}B. \, x$ cannot be
typed anymore, as the capability variable $x$ cannot be captured in
the inner-most lambda; thus preservation breaks.

The definition of the function $pure$ is changed slightly by allowing
type variables to be in the pure environment. This is natural, as we
know in the typing rule \textsc{T-TApp} that a type variable cannot be
the capability type $E$.

\subsection{Soundness}

We proved both progress and preservation of the system.

\begin{theorem}[Progress]
If $\varnothing \vdash t : T$, then either $t$ is a value or there is some
$t'$ with $t \longrightarrow t'$.
\end{theorem}

\begin{theorem}[Preservation]
If $\Gamma \vdash t : T$, and $t \longrightarrow t'$, then $\Gamma
\vdash t' : T$.
\end{theorem}

The proof of progress is the same as in System F. In the proof of
preservation, we need to make small changes to the standard
substitution lemmas in System F.

\begin{lemma}[Subsitution-Term]
  If $\Gamma,\; x:S \vdash t : T$, $s$ is a value and
  $\Gamma \vdash s : S$, then $\Gamma \vdash [x \mapsto s]t : T$.
\end{lemma}

\begin{lemma}[Subsitution-Type]
  If $\Gamma,\; X \vdash t : T$ and P $\neq$ E,
  then $\Gamma \vdash [X \mapsto P]t : [X \mapsto P]T$.
\end{lemma}

We restrict $s$ to be a value in the lemma \emph{Substitution-Term}
for the same reason as in the system STLC-Pure. In the lemma
\emph{Substitution-Type}, we restrict that $P$ is not the capability
type $E$. Otherwise, the lemma cannot be proved as explained in the
previous section.

\subsection{Effect Safety}

We follow the same approach as in the system STLC-Pure in the
formulation of effect safety.

\subsubsection{Inhabited Types and Environments}

In the presence of universal types, we need to adapt the definition of
inhabited type and inhabited environment.

The definition of inhabited environment in STLC-Pure would reject a
well-formed environment with type variables like
$\{T, \; S, \; f: T \to S, \; x:T\}$, which could be the environment
for the following well-defined function:

\begin{lstlisting}[language=Scala]
  def apply[T, S](f: T -> S)(x: T) = f x
\end{lstlisting}

To handle this problem, we propose the definition of \emph{inhabited
  type} and \emph{inhabited environment} as shown in
Figure~\ref{fig:f-pure-inhabited}.

\begin{figure}[h]
\begin{framed}

% multi-column separator
\setlength{\columnseprule}{0.4pt}
\begin{multicols}{2}

\textbf{Primitive Environment}

\infax[P-Base]
{ x:B, \; y:E \quad primitive }

\infrule[P-TVar]
{ \Sigma \quad primitive }
{ \Sigma ,\; X \quad primitive }

\infrule[P-Type]
{ \Sigma \quad primitive }
{ \Sigma ,\; x:X \quad primitive }

\textbf{Inhabited Type}

\infrule[IT]
{\Sigma \; primitive \andalso \Sigma \vdash v : T}
{ T \quad inhabited }

\columnbreak

\textbf{Inhabited Environment}

\infax[IE-Empty]
{ \varnothing \quad inhabited }

\infrule[IE-TVar]
{ \Gamma \quad inhabited }
{ \Gamma ,\; X \quad inhabited }

\infrule[IE-Type]
{ \Gamma \; inhabited \andalso T \; inhabited   }
{ \Gamma ,\; x:T \quad inhabited }

\end{multicols}
\end{framed}

\caption{System F-Pure Inhabited Environment }
\label{fig:f-pure-inhabited}
\end{figure}

The definition of inhabited type depends on \emph{primitive
  environments}. A primitive environment is an extension of the
environment $\{x:B, y:E\}$ with any type variables and type variable
bindings. This definition would take types like $X$, $X \to Y$ as
inhabited, as expected.

The definition of inhabited environment not only allow bindings of
inhabited types, but also allow any type variables to be present in an
inhabited environment. This definition ensures that only uninhabited
types, such as $B \to E$, $\forall X.X$ and
$\forall X.\forall Y.X \to Y$ are rejected from pure environments.

\subsubsection{Formulation}

As in STLC, the standard formulation is given based on inhabited
environments:

\begin{definition}[Effect-Safety-Inhabited]
  If $\Gamma$ is a pure and inhabited environment, then there doesn't
  exist $t$ with $\Gamma \vdash t : E$.
\end{definition}

The proof of the statement depends on a more general statement about
\emph{capsafe environments}. If we can arrive at such a definition of
\emph{capsafe environment} that a pure and inhabited environment is
also capsafe, then it suffices to prove the following statement about
capsafe environments:

\begin{definition}[Effect-Safety]
  If $\Gamma$ is capsafe, there doesn't exist $t$ with
  $\Gamma \vdash t : E$.
\end{definition}

What \emph{capsafe} and \emph{caprod} rules we need for universal
types? Obviously, we need to take the uninhabited type $\forall X.X$
as \emph{caprod}, as with a variable of this type, it's possible to
create a term of the capability type $E$. For example, if $x$ is of
the type $\forall X.X$ and $b$ is of the type $B$, then
$x \; [B \to E] \; b$ has the type $E$.  We also need to take the
uninhabited type $\forall X. \forall Y. X \to Y$ as
\emph{caprod}. Otherwise, if $x$ is of the type
$\forall X. \forall Y. X \to Y$ and $b$ is of the type $B$, then
$x \; [B] \; [B \to E] \; b \; b$ has the type $E$. This observation
leads us to the definition of \emph{capsafe environment} presented in
Figure~\ref{fig:f-pure-capsafe-definition}, with differences from
STLC-Pure highlighted.

\begin{figure}[h]
\begin{framed}

% multi-column separator
\setlength{\columnseprule}{0.4pt}
\begin{multicols}{2}

\textbf{Capsafe Type}

\infax[CS-Base]
{ B \quad capsafe }

\infrule[CS-Fun1]
{ S \quad caprod }
{ S \to T \quad capsafe }

\infrule[CS-Fun2]
{ T \quad capsafe }
{ S \to T \quad capsafe }

\infrule[CS-All]
{ \colorbox{shade}{$[X \mapsto B]T \; capsafe \andalso [X \mapsto E]T \; capsafe$} }
{ \colorbox{shade}{$\forall X.T \quad capsafe$} }

\columnbreak

\textbf{Caprod Type}

\infax[CP-Eff]
{ E \quad caprod }

\infrule[CP-Fun]
{ S \; \text{capsafe} \andalso T \; caprod }
{ S \to T \quad caprod }

\infrule[CP-All1]
{ \colorbox{shade}{$[X \mapsto B]T \; caprod$} }
{ \colorbox{shade}{$\forall X.T \quad caprod$} }

\infrule[CP-All2]
{ \colorbox{shade}{$[X \mapsto E]T \; caprod$} }
{ \colorbox{shade}{$\forall X.T \quad caprod$} }

\textbf{Capsafe Environment}

\infax[CE-Empty]
{ \varnothing \quad capsafe }

\infrule[CE-Var]
{ \Gamma \; capsafe \andalso T \; capsafe }
{ \Gamma, \; x:T \quad capsafe }

\infrule[CE-TVar]
{ \colorbox{shade}{$\Gamma \quad capsafe$} }
{ \colorbox{shade}{$\Gamma, \; X \quad capsafe$} }

\hfill\\

\end{multicols}
\end{framed}

\caption{System F-Pure Capsafe Environment}
\label{fig:f-pure-capsafe-definition}
\end{figure}

Why this formulation of capsafe environment is acceptable? In short,
it's because the statement \emph{Effect-Safety} logically implies the
statement \emph{Effect-Safety-Inhabited}.

The logical implication holds because a pure and inhabited
environment is also a capsafe environment. This claim has been
formally proved:

\begin{lemma}[Inhabited-Capsafe]
  If the type $T$ is inhabited, then either $T$ is capsafe or $T = E$.
\end{lemma}

\begin{theorem}[Inhabited-Pure-Capsafe-Env]
  If $\Gamma$ is pure and inhabited, then $\Gamma$ is also capsafe.
\end{theorem}

% From the theorem above, it's obvious that any property proved for a
% capsafe environment also holds for a pure and inhabited
% environment. Thus, it suffices to prove the statement
% \emph{Effect-Safety}

\subsubsection{Proof}

The proof of effect safety is more involved than in STLC-Pure. We need
to introduce the \emph{degree} of types in the proof of
relevant lemmas about types.

\begin{definition}[Degree of Type]
  The degree of a type $T$ is defined as follows:
  \begin{equation*}
    degree(T) =
    \begin{cases}
      max(degree(t_1), degree(t_2)) & \text{if } T = T_1 \to T_2,\\
      degree(T_1) + 1 & \text{if } T = \forall X.T_1,\\
      0 & others
    \end{cases}
  \end{equation*}
\end{definition}

With the help of the definition above, it's possible to prove
following lemmas based on a nested induction on the degree of types
and the type $T$.

\begin{lemma}[Capsafe-Not-Caprod]
 If type $T$ is capsafe, then $T$ is not caprod.
\end{lemma}

\begin{lemma}[Capsafe-Or-Caprod]
 For any type $T$, $T$ is either capsafe or caprod.
\end{lemma}

\begin{lemma}[Capsafe-All-Subst]
  If $\forall X.T$ is capsafe, then for any type U, $[X \mapsto U]T$
  is capsafe.
\end{lemma}

To prove the lemma \emph{Capsafe-Env-Capsafe}, we need a similar
definition on terms, and then do a nested induction on the degree of
terms and the typing relation.

\begin{definition}[Degree of Term]
  The degree of a term $t$ is defined as follows:
  \begin{equation*}
    degree(t) =
    \begin{cases}
      degree(t_1) & \text{if } t = \uplambda x{:}T. \, t_1,\\
      max(degree(t_1), degree(t_2)) & \text{if } t = t_1 \; t_2,\\
      degree(t_1) + 1 & \text{if } t = \uplambda X. \, t_1,\\
      degree(t_1) & \text{if } t = t_1 \; [T],\\
      0 & others
    \end{cases}
  \end{equation*}
\end{definition}

Effect safety follows immediately from the lemma
\emph{Capsafe-Env-Capsafe}.

\begin{lemma}[Capsafe-Env-Capsafe]
  If $\Gamma$ is capsafe and $\Gamma \vdash t : T$, then T is capsafe.
\end{lemma}

\begin{theorem}[Effect-Safety]
  If $\Gamma$ is capsafe, then there doesn't exist $t$ with
  $\Gamma \vdash t : E$.
\end{theorem}
