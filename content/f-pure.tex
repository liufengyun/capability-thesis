\section{System F-Pure}

In this chapter, we present the system \emph{F-Pure}, which is an
extension of the system STLC-Pure with universal types. In this
system, not only functions need to observe a variable-capturing
discipline, type abstractions also need to observe the same
variable-capturing discipline.

We'll first introduce the formalization, then discuss soundness and
effect safety. In the discussion, we'll focus on its difference from
the system STLC-Pure.

\begin{figure}
\begin{framed}

% multi-column separator
\setlength{\columnseprule}{0.4pt}
\begin{multicols}{2}

\textbf{Syntax}

\begin{tabu} to \linewidth {l l l X[r]}
  t   & ::= &                                      & terms:               \\
      &     &  x                                   & variable             \\
      &     & $\lambda$ x:T.t                      & abstraction          \\
      &     & t t                                  & application          \\
      &     & \colorbox{shade}{$\lambda$ X.t}      & type abstraction     \\
      &     & \colorbox{shade}{t [T]}              & type application     \\
\\
  v   & ::= &                    & values:              \\
      &     & $\lambda$ x:T.t    & abstraction value    \\
      &     & x                  & variable value       \\
      &     & \colorbox{shade}{$\lambda X.t$}    & type abstraction value  \\
\\
  T   & ::= &                       & types:               \\
      &     & \colorbox{shade}{X}   & type variable        \\
      &     & B                     & basic type           \\
      &     & E                     & capability type      \\
      &     & T $\to$ T             & type of functions    \\
      &     & \colorbox{shade}{$\forall$ X.T} & universal type       \\
\end{tabu}

\hfill\\

\textbf{Evaluation} \hfill \framebox[1.2\width][r]{$t \longrightarrow t'$}

\infrule[E-App1]
{ t_1 \longrightarrow t'_1 }
{ t_1 \; t_2 \longrightarrow t'_1 \; t_2 }

\infrule[E-App2]
{ t_2 \longrightarrow t'_2 }
{ v_1 \; t_2 \longrightarrow v_1 \; t'_2 }

\infax[E-AppAbs]
{ (\lambda x:T.t_1) v_2 \longrightarrow [x \mapsto v_2]t_1 }

\infrule[E-TApp]
{ \colorbox{shade}{$t_1 \longrightarrow t'_1$} }
{ \colorbox{shade}{$t_1 \; [T_2] \longrightarrow t'_1 \; [T_2]$} }

\infax[E-TappTabs]
{ \colorbox{shade}{$(\lambda X.t_1) [T_2] \longrightarrow [X \mapsto T_2]t_1$} }

\columnbreak

\textbf{Typing}  \hfill \framebox[1.2\width][r]{$\Gamma \vdash x : T$}

\infrule[T-Var]
{ x: T \in \Gamma }
{ \Gamma \vdash x : T }

\infrule[T-Abs]
{ pure(\Gamma),\; x: S \vdash t_2 : T }
{ \Gamma \vdash \lambda x:S.t_2 : S \to T }

\infrule[T-App]
{ \Gamma \vdash t_1 : S \to T \andalso \Gamma \vdash t_2 : S }
{ \Gamma \vdash t_1 \; t_2 : T }

\infrule[T-TAbs]
{ \colorbox{shade}{$pure(\Gamma),\; X \vdash t_2 : T$} }
{ \colorbox{shade}{$\Gamma \vdash \lambda X.t_2 : \forall X. T$} }

\infrule[T-TApp]
{ \colorbox{shade}{$\Gamma \vdash t_1 : \forall X.T \andalso T_2 \neq E$} }
{ \colorbox{shade}{$\Gamma \vdash t_1 \; [T_2] : [X \mapsto T_2]T$} }

\hfill\\

\textbf{Pure Environment}

\hfill

\begin{center}
\begin{tabular}{l c l}
pure($\varnothing$)             & = &   $\varnothing$ \\
pure($\Gamma$, x: E)            & = &  pure($\Gamma$) \\
pure($\Gamma$, x: T)  & = &  pure($\Gamma$), x: T     \\
\rowcolor{gray!40}
pure($\Gamma$, X)  & = &  pure($\Gamma$), X  \\
\end{tabular}
\end{center}


\end{multicols}
\end{framed}

\caption{System F-Pure}
\label{fig:f-pure-definition}
\end{figure}

\subsection{Definitions}

The system F-Pure extends STLC-Pure with universal
types. Figure~\ref{fig:f-pure-definition} presents the full definition
of F-Pure, with the difference from the system STLC-Pure highlighted.

The extension of syntax and evaluation rules are exactly the same as
the extension of standard STLC with universal types.  The essential
difference lies in the two new typing rules \textsc{T-TAbs} and
\textsc{T-TApp}. The typing rule \textsc{T-TAbs} stipulates that type
abstraction must observe the variable-capturing discipline.

\infrule[T-TAbs]
{ pure(\Gamma),\; X \vdash t_2 : T }
{ \Gamma \vdash \lambda X.t_2 : \forall X. T }

We made this design choice in order to allow universal types to be
present in pure environments. Otherwise, if type abstractions can
capture capability variables, application of a type abstraction could
generate a term of the capability type or have side effects. This
makes it incorrect to have universal types in pure environments, thus
renders universal types useless in the system.

The typing rule \textsc{T-TApp} requires that the type parameter
cannot be the capability type E. However, it's allowed to supply
non-inhabitable types like $B \to E$ as parameter to type abstraction.

% This restriction implies in system F-Pure polymorphism doesn't cover
% capability types.

\infrule[T-TApp]
{ \Gamma \vdash t_1 : \forall X.T \andalso T_2 \neq E }
{ \Gamma \vdash t_1 \; [T_2] : [X \mapsto T_2]T }

Without the restriction, preservation of the system breaks, as can be
seen from following term $t$, which has the type
$\forall T. T \to B \to T$:

\begin{center}
  $t = \lambda T. \; \lambda x:T. \; \lambda y:B. \; x$
\end{center}

If we allow $E$ as parameter to type application, the term $t [E]$ has
the type $E \to B \to E$. However, after one evaluation step, the term
$\lambda x:E. \; \lambda y:B. \; x$ cannot be typed anymore, as the
capability variable $x$ cannot be captured in the inner-most lambda;
thus preservation breaks.

The definition of the function \emph{pure} is changed slightly by
allowing type variables to be in the pure environment. This is
natural, as we know in the typing rule \textsc{T-TApp} that a type
variable cannot be of the capability type.

\subsection{Soundness}

We proved both progress and preservation of the system.

\begin{theorem}[Progress]
If $\varnothing \vdash t : T$, then either $t$ is a value or there is some
$t'$ with $t \longrightarrow t'$.
\end{theorem}

\begin{theorem}[Preservation]
If $\Gamma \vdash t : T$, and $t \longrightarrow t'$, then $\Gamma
\vdash t' : T$.
\end{theorem}

The proof of progress is the same as in System F. In the proof of
preservation, we need to make small changes to the standard
substitution lemmas in System F.

\begin{lemma}[Subsitution-Term]
  If $\Gamma,\; x:S \vdash t : T$, s is a value and
  $\Gamma \vdash s : S$, then $\Gamma \vdash [x \mapsto s]t : T$.
\end{lemma}

\begin{lemma}[Subsitution-Type]
  If $\Gamma,\; X \vdash t : T$ and P $\neq$ E,
  then $\Gamma \vdash [X \mapsto P]t : [X \mapsto P]T$.
\end{lemma}

We restrict $s$ to be a value in the lemma \emph{Substitution-Term}
for the same reason as in the system STLC-Pure. In the lemma
\emph{Substitution-Type}, we restrict that $P$ is not the capability
type E. Otherwise, the lemma cannot be proved as explained in the
previous section.

\subsection{Effect Safety}

We follow the same approach as in the system STLC-Pure in the
formulation of effect safety. The formulation is an extension of the
definition of \emph{capsafe} and \emph{caprod} in STLC-Pure with
universal types.

\subsubsection{Formulation}

As in STLC, the standard formulation is given based on inhabitable
environments:

\begin{definition}[Effect-Safety-Inhabitability]
  If $\Gamma$ is a pure and inhabitable environment, then there
  doesn't exist $t$ with $\Gamma \vdash t : E$.
\end{definition}

The proof of the statement depends on a weaker and more general
statement about \emph{healthy environments}. If we can arrive at such
a definition of \emph{healthy environment} that a pure and inhabitable
environment is also healthy, then it suffices to prove the following
statement about healthy environments:

\begin{definition}[Effect-Safety]
  If $\Gamma$ is healthy, there doesn't exist $t$ with
  $\Gamma \vdash t : E$.
\end{definition}

What \emph{capsafe} and \emph{caprod} rules we need for universal
types? Obviously, we need to take the non-inhabitable type
$\forall X.X$ as \emph{caprod}, as with a variable of this type, it's
possible to create a term of the capability type E. For example, if
$x$ is of the type $\forall X.X$ and $b$ is of the type $B$, then
$x \; [B \to E] \; b$ has the type $E$.  We also need to take the
non-inhabitable type $\forall X. \forall Y. X \to Y$ as
\emph{caprod}. Otherwise, if $x$ is of the type
$\forall X. \forall Y. X \to Y$ and $b$ is of the type $B$, then
$x \; [B] \; [B \to E] \; b \; b$ has the type $E$. This observation
leads us to following rules for universal types.

\infrule[CS-All]
{ [X \mapsto B]T \; \text{capsafe} \andalso [X \mapsto E]T \; \text{capsafe} }
{ \forall X.T \quad \text{capsafe} }

\infrule[CP-All]
{ [X \mapsto B]T \; caprod \quad or \quad [X \mapsto E]T \; caprod }
{ \forall X.T \quad caprod }

% We need to ensure that thew new rule CP-All only mark non-inhabitable
% types as $caprod$. As before, we only provide informal argument
% here. The justification for the rule \text{CP-All} is that, if a
% universal type is inhabitable, then it must be inhabitable by
% replacing the type parameter with any type (except E). Thus if a
% specialized universal type is caprod (which we know is non-inhabitable
% as argued before), then the universal type is also
% non-inhabitable. Therefore, the \textsc{CP-All} rule only marks
% non-inhabitable types as caprod.

% Note that in the explanation above, we say that the capability type E
% is an exception. For example, the type $\forall T.T \to B \to T$ is
% inhabited by the term
% $\lambda T. \; \lambda x:T. \; \lambda y:B. \; x$. However, the type
% $E \to B \to E$ is non-inhabitable. This is not a problem since
% $E \to B \to E$ is \emph{capsafe}. In fact, in the initial
% formulation, we used $B \to E$ instead of $E$ in \textsc{CS-All} and
% \textsc{CP-All}. Later, we found out that we can simplify the
% formulation without changing the proof of effect safety.

The full definition of the \emph{healthy environment} is presented in
Figure~\ref{fig:f-pure-healthy-definition}, with difference from
STLC-Pure highlighted.

\begin{figure}[h]
\begin{framed}

% multi-column separator
\setlength{\columnseprule}{0.4pt}
\begin{multicols}{2}

\textbf{Capsafe}

\infax[CS-Base]
{ B \quad \text{capsafe} }

\infrule[CS-Fun1]
{ S \quad caprod }
{ S \to T \quad \text{capsafe} }

\infrule[CS-Fun2]
{ T \quad \text{capsafe} }
{ S \to T \quad \text{capsafe} }

\infrule[CS-All]
{ \colorbox{shade}{$[X \mapsto B]T \; \text{capsafe} \andalso [X \mapsto E]T \; \text{capsafe}$} }
{ \colorbox{shade}{$\forall X.T \quad \text{capsafe}$} }

\columnbreak

\textbf{Caprod}

\infax[CP-Eff]
{ E \quad caprod }

\infrule[CP-Fun]
{ S \; \text{capsafe} \andalso T \; caprod }
{ S \to T \quad caprod }

\infrule[CP-All]
{ \colorbox{shade}{$[X \mapsto B]T \; caprod \quad or \quad [X \mapsto E]T \; caprod$} }
{ \colorbox{shade}{$\forall X.T \quad caprod$} }

\textbf{Healthy}

\infax[H-Empty]
{ \varnothing \quad caprod }

\infrule[H-Var]
{ G \; healthy \andalso T \; \text{capsafe} }
{ G, \; x:T \quad healthy }

\infrule[H-TVar]
{ \colorbox{shade}{$G \quad healthy$} }
{ \colorbox{shade}{$G, \; X \quad healthy$} }

\hfill\\

\end{multicols}
\end{framed}

\caption{System F-Pure Healthy Environment}
\label{fig:f-pure-healthy-definition}
\end{figure}

Why this formulation of healthy environment is acceptable? In short,
it's because the statement \emph{Effect-Safety} logically implies the
statement \emph{Effect-Safety-Inhabitability}.

The logical implication holds because a pure and inhabitable
environment is also a healthy environment. This claim has been
formally proved:

\begin{theorem}[Inhabitable-Capsafe]
  If the type T is inhabitable, then either T is capsafe or $T = E$.
\end{theorem}

\begin{theorem}[Inhabitable-Pure-Healthy]
  If $\Gamma$ is pure and inhabitable, then $\Gamma$ is also healthy.
\end{theorem}

% From the theorem above, it's obvious that any property proved for a
% healthy environment also holds for a pure and inhabitable
% environment. Thus, it suffices to prove the statement
% \emph{Effect-Safety}

\subsubsection{Proof}

The proof of effect safety is more involved than in STLC-Pure. We need
to introduce the \emph{degree} of types in the proof of
relevant lemmas about types.

\begin{definition}[Degree of Type]
  The degree of a type $T$ is defined as follows:
  \begin{equation*}
    degree(T) =
    \begin{cases}
      max(degree(t_1), degree(t_2)) & \text{if } T = T_1 \to T_2,\\
      degree(T_1) + 1 & \text{if } T = \forall X.T_1,\\
      0 & others
    \end{cases}
  \end{equation*}
\end{definition}

With the help of the definition above, it's possible to prove
following lemmas based on double induction on the degree of types and
the type T.

\begin{lemma}[Capsafe-Not-Caprod]
 If type T is capsafe, then T is not caprod.
\end{lemma}

\begin{lemma}[Capsafe-Or-Caprod]
 For any type T, T is either capsafe or caprod.
\end{lemma}

% \begin{lemma}[Healthy-Pure]
%   If the environment $\Gamma$ is healthy, then $pure \; \Gamma = \Gamma$.
% \end{lemma}

\begin{lemma}[Capsafe-All-Subst]
  If $\forall X.T$ is capsafe, then for all type U, $[X \mapsto U]T$
  is capsafe.
\end{lemma}

To prove the lemma \emph{Healthy-Capsafe}, we need a similar
definition on terms, and then do a double induction on the degree of
terms and the typing relation. Effect safety follows immediately from
the lemma \emph{Healthy-Capsafe}.

\begin{definition}[Degree of Term]
  The degree of a term $t$ is defined as follows:
  \begin{equation*}
    degree(t) =
    \begin{cases}
      degree(t_1) & \text{if } t = \lambda x:T.t_1,\\
      max(degree(t_1), degree(t_2)) & \text{if } t = t_1 \; t_2,\\
      degree(t_1) + 1 & \text{if } t = \lambda X.t_1,\\
      degree(t_1) & \text{if } t = t_1 \; [T],\\
      0 & others
    \end{cases}
  \end{equation*}
\end{definition}

\begin{lemma}[Healthy-Capsafe]
  If $\Gamma$ is healthy and $\Gamma \vdash t : T$, then T is capsafe.
\end{lemma}

\begin{theorem}[Effect-Safety]
  If $\Gamma$ is healthy, then there doesn't exist $t$ with
  $\Gamma \vdash t : E$.
\end{theorem}
