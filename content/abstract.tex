\begin{abstract}
  Type-and-effect systems have been around for about thirty years. But
  they have never gained popularity in the programmer community. One
  cause might be the verbosity of its syntax. \emph{Monad-based}
  effect systems have a successful story in Haskell, but these systems
  can't handle \emph{effect polymorphism} well. Haskell programmers
  often need to maintain two versions of the same code, one is pure,
  the other is impure.

  In this study we took another approach and studied
  \emph{capability-based} effect systems. Generally, capability-based
  effect systems are easier to understand and use than monads. Most
  importantly, powered by the combination of \emph{stoic functions}
  and \emph{free functions}, capability-based systems can quite well
  handle effect polymorphism. These merits make capability-based
  effect systems stand a better chance to be accepted by the
  programmer community.

  In this report we present four capability-based type-and-effect
  systems of increasing complexity, namely STLC-Pure, F-Pure,
  STLC-Impure and F-Impure. We proved soundness and effect safety for
  all the systems based on locally-nameless representation in Coq.

  % The system SLTC-Pure and F-Pure demonstrated that capability-based
  % effect systems can be used in pure functional languages to track
  % effects. The system STLC-Impure and F-Impure demonstrated that
  % capability-based effect systems can be used in a hybrid language
  % where only part of the program is effect-disciplined -- this
  % provides a flexibility to programmers when they don't care about
  % effects.

\end{abstract}
