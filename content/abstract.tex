\section*{\centering Abstract}
\addcontentsline{toc}{section}{Abstract}

Type-and-effect systems have been around for about thirty years. But
they have never gained popularity in the programming community, mainly
due to the verbosity of its syntax. \emph{Monad-based} effect systems
have a successful story in Haskell, but these systems can't handle
\emph{effect polymorphism} well.  Haskell programmers often need to
maintain both a monadic and non-monadic version of the same code.

In this study we took another approach and investigated
\emph{capability-based} effect systems. In such systems, an instance
of capability is required to make side effects, and capabilities are
passed via function parameters. Thus, by tracking capabilities in the
type system, we can track effects in the program. To ensure
capabilities are passed as function parameters instead of being
captured from the environment, we introduced \emph{stoic functions}
which observe a \emph{variable-capturing discipline} , in contrast to
\emph{free functions}, which have no constraints in capturing
variables.

Generally speaking, capability-based effect systems are easier to
understand and use than monad-based effect systems. More importantly,
equipped with stoic functions and free functions, capability-based
systems can handle \emph{effect polymorphism} elegantly. These merits
make capability-based effect systems stand a better chance to be
adopted by the programming community.

In this report, we present four capability-based type-and-effect
systems of increasing complexity, namely STLC-Pure, STLC-Impure,
F-Pure and F-Impure. STLC-Pure is a variant of standard STLC with only
stoic functions, and the latter three is a gradual enrichment of
STLC-Pure with free functions, subtyping and universal types. We
discuss the subtleties of the meta-theories of each system. We also
show how to solve the problem of effect polymorphism with the
combination of free functions and stoic functions.
