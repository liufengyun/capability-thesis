\section*{\centering Abstract}
\addcontentsline{toc}{section}{Abstract}

The problem of \emph{effect polymorphism} is the main obstacle to the
wide adoption of effect systems in the programming community. The
absence of effect systems reduces compiler optimization opportunities
and deprives programmers of the freedom to impose effect constraints
on APIs in parallel and distributed computations.

This study shows that \emph{capability-based} effect systems, equipped
with \emph{stoic functions} and \emph{free functions}, can easily
solve the problem of \emph{effect polymorphism} without incurring
notational burden on programmers. This advantage makes
capability-based effect systems stand a better chance to be adopted by
the programming community.

The central idea of \emph{capability-based} effect system is that an
instance of capability is required in order to make side effects. If
capabilities are passed as function parameters, by tracking
capabilities in the type system we can track effects in the program.

To ensure that capabilities are passed through function parameters,
instead of being captured from the environment, we need to impose a
\emph{variable-capturing discipline}, stipulating that capability
variables cannot be captured. Functions observe the discipline are
called \emph{stoic functions}, while functions don't observe the
discipline are called \emph{free functions}.
